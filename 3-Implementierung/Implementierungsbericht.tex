\documentclass[parskip=full]{scrartcl}
\usepackage[T1]{fontenc}
\usepackage[utf8]{inputenc}
\usepackage[ngerman]{babel}
\usepackage{hyperref}
\hypersetup{
	pdftitle={PSE: Blockchain-basiertes E-Voting - Implementierungsbericht},%
	,%
}
\usepackage{graphicx}
\usepackage{csquotes}
\usepackage[nonumberlist]{glossaries}
\usepackage{enumitem}
\usepackage{xcolor}
\usepackage{svg}
\usepackage[section]{placeins}

\makeatletter
\AtBeginDocument{%
	\expandafter\renewcommand\expandafter\subsection\expandafter{%
		\expandafter\@fb@secFB\subsection
	}%
}
\makeatother
\makeatletter
\AtBeginDocument{%
	\expandafter\renewcommand\expandafter\subsubsection\expandafter{%
		\expandafter\@fb@secFB\subsubsection
	}%
}
\makeatother

\addto\extrasngerman{\def\figureautorefname{Abb.}}
\newcommand{\textitx}[1]{\mbox{\textit{#1}}}
\newcommand{\fakeparagraph}[1]{\textbf{#1}}
%\renewcommand{\includesvg}[1][1]{}


\title{
	PSE:Blockchain-basiertes E-Voting \\
	Implementierungsbericht
}
\author{Tim Fröhlich, Achim Kriso, Philipp Schaback, David Schuldes, Artem Vasilev\\ Phasenverantwortlicher: David Schuldes}



\begin{document}
\clearpage
\maketitle
\pagenumbering{gobble}
\newpage

\tableofcontents
\newpage
\pagenumbering{arabic}

\section{Einleitung}
Dieses Dokument beschreibt die Ergebnisse der Implementierungsphase, die
im Rahmen des Moduls Praxis der Softwareentwicklung (PSE) am Lehrstuhl „Anwendungsorientierte formale Verifikation" von Prof. Dr. Beckert am Karlsruher Institut für
Technologie entstanden sind.
Implementiert wurde die im Pflichtenheft vorgegebene und in der Entwurfsphase entworfene Software "Blockchain-basiertes Evoting".


\section{Zeitablauf}


\section{Umsetzung der funktionalen Anforderungen}

\subsection{Musskriterien}

\subsection{Sollkriterien}

\subsection{Kannkriterien}


\section{Umsetzung der nichtfunktionalen-Anforderungen}

Alle im Pflichtenheft definierten, nicht-funktionale Anforderung wurden erfüllt. Im Folgenden sind sie in die Kategorien Zeitverhalten und Benutzerfreundlichkeit unterteilt.

\subsection{Zeitverhalten}

\begin{itemize}
	\item Die Stimmabgabe eines Wählers dauert vorraussichtlich nicht länger als 5 Minuten.
	\item Die grafische Benutzeroberfläche reagiert sofort entsprechend der Aktion des Benutzers.
	\item Die Auswertung einer Wahl nimmt bei höchstens 10000 abgegebenen Stimmen vorraussichtlich nicht mehr als 5 Minuten in Anspruch.
\end{itemize}

\subsection{Benutzerfreundlichkeit}
\begin{itemize}
	\item Die Benutzung der Software erfordert keine besonderen Vorkenntnisse, so dass sie auch Benutzer mit geringen Computerkenntnissen verwenden können.
\end{itemize}
		
\section{Umsetzung von Entwurfsentscheidungen}


\section{Änderungen zum Entwurf}


\end{document}

