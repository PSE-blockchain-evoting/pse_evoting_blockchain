\documentclass[parskip=full]{scrartcl}
\usepackage[T1]{fontenc}
\usepackage[utf8]{inputenc}
\usepackage[ngerman]{babel}
\usepackage{hyperref}
\hypersetup{
	pdftitle={PSE: Blockchain-basiertes E-Voting - Implementierungsbericht},%
	,%
}
\usepackage{graphicx}
\usepackage{csquotes}
\usepackage[nonumberlist]{glossaries}
\usepackage{enumitem}
\usepackage{xcolor}
\usepackage{svg}
\usepackage[section]{placeins}

\makeatletter
\AtBeginDocument{%
	\expandafter\renewcommand\expandafter\subsection\expandafter{%
		\expandafter\@fb@secFB\subsection
	}%
}
\makeatother
\makeatletter
\AtBeginDocument{%
	\expandafter\renewcommand\expandafter\subsubsection\expandafter{%
		\expandafter\@fb@secFB\subsubsection
	}%
}
\makeatother

\addto\extrasngerman{\def\figureautorefname{Abb.}}
\newcommand{\textitx}[1]{\mbox{\textit{#1}}}
\newcommand{\fakeparagraph}[1]{\textbf{#1}}
%\renewcommand{\includesvg}[1][1]{}


\title{
	PSE:Blockchain-basiertes E-Voting \\
	Implementierungsbericht
}
\author{Tim Fröhlich, Achim Kriso, Philipp Schaback, David Schuldes, Artem Vasilev\\ Phasenverantwortlicher: Philipp Schaback}



\begin{document}
\clearpage
\maketitle
\pagenumbering{gobble}
\newpage

\tableofcontents
\newpage
\pagenumbering{arabic}

\section{Einleitung}
Dieses Dokument beschreibt die Ergebnisse der Implementierungsphase, die
im Rahmen des Moduls Praxis der Softwareentwicklung (PSE) am Lehrstuhl „Anwendungsorientierte formale Verifikation" von Prof. Dr. Beckert am Karlsruher Institut für
Technologie entstanden sind.
Implementiert wurde die im Pflichtenheft vorgegebene und in der Entwurfsphase entworfene Software "Blockchain-basiertes Evoting".


\section{Zeitablauf}


\section{Umsetzung der funktionalen Anforderungen}

\subsection{Musskriterien}
Alle im Pflichtenheft festgelegten Musskriterien wurden durch die Implementierung umgesetzt.

\subsection{Sollkriterien}
Die Sollkriterien wurden alle bis auf eines erfüllt:
\textit{S3: Dynamische Peerverbindung}\\
Teil des Sicherheitsmodells von Hyperledger Fabric ist, das ein Client Verbindungen zu mehreren Peers herstellt, das Kriterium sieht aber nur die Verbindung zu einem Peer vor.
Somit wäre eine Umsetzung von S3 der korrekten und sicheren Wahl entgegenstehend. Dennoch ermöglichen wir es, einzelne Peers auszuschließen, mittels der Konfigurationsdatei. 

\subsection{Kannkriterien}
Alle Kannkriterien, ausgenommen von \textit{K2: Geheime Wahlen} wurden umgesetzt. K2 wurde, wie schon im Entwurfsdokument begründet, nicht implementiert.

\section{Umsetzung der nichtfunktionalen-Anforderungen}

\subsection{Produktleistungen}
	
\subsection{Weitere nichtfunktionale Anforderungen}

		
\section{Umsetzung von Entwurfsentscheidungen}


\section{Änderungen zum Entwurf}


\end{document}

