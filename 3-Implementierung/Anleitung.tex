\documentclass[parskip=full]{scrartcl}
\usepackage[T1]{fontenc}
\usepackage[utf8]{inputenc}
\usepackage[ngerman]{babel}
\usepackage{hyperref}
\hypersetup{
	pdftitle={PSE: Blockchain-basiertes E-Voting - Anleitung},%
	,%
}
\usepackage{graphicx}
\usepackage{csquotes}
\usepackage[nonumberlist]{glossaries}
\usepackage{enumitem}
\usepackage{xcolor}
\usepackage{svg}
\usepackage[section]{placeins}

\makeatletter
\AtBeginDocument{%
	\expandafter\renewcommand\expandafter\subsection\expandafter{%
		\expandafter\@fb@secFB\subsection
	}%
}
\makeatother
\makeatletter
\AtBeginDocument{%
	\expandafter\renewcommand\expandafter\subsubsection\expandafter{%
		\expandafter\@fb@secFB\subsubsection
	}%
}
\makeatother

\addto\extrasngerman{\def\figureautorefname{Abb.}}
\newcommand{\textitx}[1]{\mbox{\textit{#1}}}
\newcommand{\fakeparagraph}[1]{\textbf{#1}}
%\renewcommand{\includesvg}[1][1]{}


\title{
	PSE:Blockchain-basiertes E-Voting \\
	Anleitung
}
\author{Tim Fröhlich, Achim Kriso, Philipp Schaback, David Schuldes, Artem Vasilev\\ Phasenverantwortlicher: Philipp Schaback}



\begin{document}
	\clearpage
	\maketitle
	\pagenumbering{gobble}
	\newpage
	
	\tableofcontents
	\pagenumbering{arabic}
	\newpage
	
	\section{Lieferumfang}
	\subsection{Artefakte}
	Im Lieferumfang enthalten sind folgende Artefakte:
	\begin{itemize}
		\item Eine ZIP-Datei \textit{network.zip}, sie enthält:\\
		Ein Bash-Skript \textit{start.sh}: Dieses Skript muss ausgeführt werden, um das Netzwerk zu starten. Wichtig ist, dass dem Skript Rechte zum Erstellen, Laden und Löschen von Docker-Containern zur Verfügung stehen. Meistens ist eine Ausführung als \textit{root} nötig.\\
		Einen Ordner \textit{votingchaincode}: Dieser beinhaltet die SmartContracts, speziell die Datei \textit{chaincode\_vote.go}.
		\item Eine ZIP-Datei \textit{client.zip}, die den Sourcecode der Klienten enthält, aus dem die .jar Dateien für den Wähler und den Wahlleiter generiert werden.
	\end{itemize}

	\subsection{Source Code}
	Der Sourcecode befindet sich in der ZIP-Datei \textit{client.zip}, im Unterordner \textit{src}.
	
	\section{Voraussetzungen}
	Zum Ausführen des Netzwerks werden Docker (getestet mit Version 18.05), Docker-Compose (getestet mit Version 1.22) und Golang (getestet mit Version 1.10) benötigt.\\
	Zum Ausführen der Klienten ist ein JRE mit Version $\ge$ 8 Voraussetzung.
	\section{Erstellen der Artefakte (optional) und Installation}
	\subsection{Klient}
	Zum Erstellen der Artefakte muss folgendermaßen vorgegangen werden:
	\begin{enumerate}
		\item \textit{client.zip} entpacken
		\item Im resultierenden Ordner den Befehl \textitx{maven package} ausführen.
	\end{enumerate}
	Die resultierenden Artefakte heißen \textit{supervisor-jar-with-dependencies.jar} (für den Wahlleiter) und \textit{voter-jar-with-dependencies.jar} (für den Wähler) und finden sich im Unterordner \textit{target}.
	\subsection{Netzwerk}
	Die ZIP-Datei \textit{network.zip} muss in einen beliebigen Ordner entpackt werden. 
	
	\section{Konfiguration}
	Die Standardkonfiguration sieht zu Testzwecken eine Ausführung des Netzwerks und der Klienten auf dem selben Computer vor. Möchte man die mitgelieferte Konfiguration ändern geht man folgendermaßen vor:
	\begin{enumerate}
		\item Eine Kopie von \textit{src/main/resources/config.properties} im Verzeichnis neben der *.jar ablegen
		\item In der Datei die URL-Pfade entsprechend anpassen
		\item EVote lädt automatisch die Konfigurationsdatei
	\end{enumerate}
	Auf Netzwerkseite ist keine weitere Konfiguration nötig, solange die Ports die in der \textit{docker-compose.yml}-Datei vom Klienten aus erreichbar sind.

	\section{Ausführen}
	\subsection{Netzwerk starten}
	Zum Starten des Netzwerkes muss das \textit{start.sh}-Skript im Netzwerkordner ausgeführt werden. Es lädt die benötigten Docker-Images und startet das vorkonfigurierte Netzwerk.
	\subsection{Client starten}
	Der Klient des Wahlleiters bzw. Wählers kann über den Kommandozeilenbefehl mit dem Befehl \textit{java -jar <Jar-File>} gestartet werden. Daraufhin sollte der Startbildschirm der jeweiligen Benutzeroberfläche angezeigt. \\
	Benutzername und Passwort für den Wahlleiter lauten \textit{admin-org1} und \textit{admin-org1pw}.

    \subsection{Generierung der Zertifikate}
    Sobald der Wahlleiter eine Wahl gestartet hat, werden Zertifikate zur eindeutigen Identifizierung aller wahlberechtigten Wähler und des Wahlleiters vom Blockchain-Netzwerk generiert. Die Wählerzertifikate werden im Unterorder \textit{VoterCertificates} des Ausführungspfades abgespeichert, das Zerifikat des Wahlleiters als \textit{ElectionSupervisorCertificate}.
    
    \section{Neustarten / Beenden}
    Zum Neustarten des Netzwerkes kann einfach erneut die Datei \textit{start.sh} ausgeführt werden,
    zum Beenden des Netzwerks die Datei \textit{stop.sh}.
		
\end{document}
