\documentclass[parskip=full]{scrartcl}
\usepackage[T1]{fontenc}
\usepackage[utf8]{inputenc}
\usepackage[ngerman]{babel}
\usepackage{hyperref}
\hypersetup{
	pdftitle={PSE: Blockchain-basiertes E-Voting - Anleitung},%
	,%
}
\usepackage{graphicx}
\usepackage{csquotes}
\usepackage[nonumberlist]{glossaries}
\usepackage{enumitem}
\usepackage{xcolor}
\usepackage{svg}
\usepackage[section]{placeins}

\makeatletter
\AtBeginDocument{%
	\expandafter\renewcommand\expandafter\subsection\expandafter{%
		\expandafter\@fb@secFB\subsection
	}%
}
\makeatother
\makeatletter
\AtBeginDocument{%
	\expandafter\renewcommand\expandafter\subsubsection\expandafter{%
		\expandafter\@fb@secFB\subsubsection
	}%
}
\makeatother

\addto\extrasngerman{\def\figureautorefname{Abb.}}
\newcommand{\textitx}[1]{\mbox{\textit{#1}}}
\newcommand{\fakeparagraph}[1]{\textbf{#1}}
%\renewcommand{\includesvg}[1][1]{}


\title{
	PSE:Blockchain-basiertes E-Voting \\
	Anleitung
}
\author{Tim Fröhlich, Achim Kriso, Philipp Schaback, David Schuldes, Artem Vasilev\\ Phasenverantwortlicher: Philipp Schaback}



\begin{document}
	\clearpage
	\maketitle
	\pagenumbering{gobble}
	\newpage
	
	\tableofcontents
	\pagenumbering{arabic}
	\newpage
	
	\section{Lieferumfang}
	\subsection{Artefakte}
	Im Lieferumfang enthalten sind folgende Artefakte:
	\begin{enumerate}
		\item Eine ZIP-Datei "network.zip", sie enthält ein bash-Skript "startFabric.sh". Diese Skript muss ausgeführt werden, um das Netzwerk zu starten. Wichtig ist, dass Rechte zum erstellen, laden und löschen von Docker-Containern vorhanden sind. Meistens ist eine Ausführung als root nötig.
		\item Einen Ordner "votingchaincode". Dieser beinhaltet die Smartcontracts, speziell die Datei "chaincode\_vote.go".
		\item Eine ZIP-Datei "client.zip", die den Sourcecode der Klienten enthält, aus dem die .jar Dateien für die beiden Klienten Voter- und Supervisor generiert werden.

	\end{enumerate}

	\subsection{Source Code}
	Der Sourcecode befindet sich in der ZIP-Datei "client.zip". Diese befindet sich in dem Unterordner "src".
	
	\section{Erstellen der Artefakte (optional) und Installation}
	\subsection{Client}
	Die client.zip-Datei wird entpackt, in dem entpackten Ordner muss der Befehl "maven package" über die Kommandozeile ausgeführt werden. Es werden zwei .jar-Dateien generiert: "supervisor-evote.jar" und "voter-evote.jar", sie entsprechen den ausführbaren Klienten für Wahlleiter und Wähler.
	\subsection{Netzwerk}
	Die ZIP-Datei "network.zip" wird entpackt, in dem entpackten Ordner befindet sich das "startFabric.sh" Skript, welches das Blockchain-Netzwerk aufsetzt. Hierfür werden docker (getestet mit Version 18.05), docker-compose (getestet mit Version 1.22) und golang (getestet mit Version 1.10) benötigt.
	Da das Skript docker und docker-compose Befehle benutzt, werden zur Ausführung entsprechende Benutzerrechte notwendig.
	
	\section{Ausführen}
	\subsection{Netzwerk starten}
	Zum Starten des Netzwerkes muss das "startFabric.sh"-Skript, welches sich im entpackten "network.zip"-Ordner befindet, ausgeführt werden.
	
	\subsection{Client starten}
	Der Klient des Wahlleiters bzw. Wählers kann über den Kommandozeilenbefehl "java -jar supervisor-evote.jar" bzw. "java -jar voter-evote.jar" gestartet werden. Daraufhin sollte der Startbildschirm der jeweiligen Benutzeroberfläche angezeigt. Benutzername und Passwort für den Wahlleiter lauten "admin" und "adminpw".
	
	\subsection{Generierung der Zertifikate}
	Sobald der Wahlleiter eine Wahl gestartet hat, werden Zertifikate zur eindeutigen Identifizierung aller wahlberechtigten Wähler und des Wahlleiters vom Blockchain-Netzwerk generiert. Sie werden im working-directory des Prozesses abgespeichert. 
		
\end{document}