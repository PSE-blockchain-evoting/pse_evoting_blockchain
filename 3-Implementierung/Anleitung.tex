\documentclass[parskip=full]{scrartcl}
\usepackage[T1]{fontenc}
\usepackage[utf8]{inputenc}
\usepackage[ngerman]{babel}
\usepackage{hyperref}
\hypersetup{
	pdftitle={PSE: Blockchain-basiertes E-Voting - Anleitung},%
	,%
}
\usepackage{graphicx}
\usepackage{csquotes}
\usepackage[nonumberlist]{glossaries}
\usepackage{enumitem}
\usepackage{xcolor}
\usepackage{svg}
\usepackage[section]{placeins}

\makeatletter
\AtBeginDocument{%
	\expandafter\renewcommand\expandafter\subsection\expandafter{%
		\expandafter\@fb@secFB\subsection
	}%
}
\makeatother
\makeatletter
\AtBeginDocument{%
	\expandafter\renewcommand\expandafter\subsubsection\expandafter{%
		\expandafter\@fb@secFB\subsubsection
	}%
}
\makeatother

\addto\extrasngerman{\def\figureautorefname{Abb.}}
\newcommand{\textitx}[1]{\mbox{\textit{#1}}}
\newcommand{\fakeparagraph}[1]{\textbf{#1}}
%\renewcommand{\includesvg}[1][1]{}


\title{
	PSE:Blockchain-basiertes E-Voting \\
	Anleitung
}
\author{Tim Fröhlich, Achim Kriso, Philipp Schaback, David Schuldes, Artem Vasilev\\ Phasenverantwortlicher: Philipp Schaback}



\begin{document}
	\clearpage
	\maketitle
	\pagenumbering{gobble}
	\newpage
	
	\tableofcontents
	\pagenumbering{arabic}
	\newpage
	
	\section{Lieferumfang}
	\subsection{Artefakte}
	Im Lieferumfang enthalten sind folgende Artefakte:
	\begin{enumerate}
		\item Eine ZIP-Datei "network.zip", sie enthält ein bash-Skript "run.sh". Dieses Skript muss ausgeführt werden, um das Netzwerk zu starten. Wichtig ist, dass Rechte zum erstellen, laden und löschen von Docker-Containern vorhanden sind. Meistens ist eine Ausführung als root nötig.
		\item Einen Ordner "votingchaincode". Dieser beinhaltet die Smartcontracts, speziell die Datei "chaincode_vote.go".
		\item Eine ZIP-Datei "client.zip", die den Sourcecode der Klienten enthält, aus dem die .jar Dateien für die beiden Klienten Voter- und Supervisor generiert werden.
		\item 
	\end{enumerate}

	\subsection{Source Code}
	
	
	\section{Erstellen der Artefakte (optional)}
	\subsection{Client}
	
	\subsection{Netzwerk}
	
	
	\section{Installieren}
	\subsection{Client}
	
	
	\subsection{Netzwerk}
	
	\section{Ausführen}
	\subsection{Netzwerk starten}
	
	\subsection{Client starten}
		
\end{document}
