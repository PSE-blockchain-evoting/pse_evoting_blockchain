\documentclass[parskip=full]{scrartcl}
\usepackage[T1]{fontenc}
\usepackage[utf8]{inputenc}
\usepackage[german]{babel}
\usepackage{hyperref}
\hypersetup{
pdftitle={PSE: Blockchain-basiertes E-Voting - Entwurfsdokument},%
,%
}
\usepackage{graphicx}
\usepackage{csquotes}
\usepackage[nonumberlist]{glossaries}
\usepackage{enumitem}
\usepackage{xcolor}
\usepackage{lscape}

\title{PSE: Blockchain-basiertes E-Voting}

\begin{document}
	\clearpage
	\maketitle
	\pagenumbering{gobble}
	\newpage
	
	\tableofcontents
	\newpage
	\pagenumbering{arabic}
	\section{Einleitung}
	
	\section{View}
		\subsection{Components}
		Diese Unterpaket stellt dem Rest des View Pakets Diagramme und ein modulares Tabellen System bereit.
		\subsubsection{Diagramme}
		Diagramme werden von der View benötigt um die Ergebnisse einer Wahl in einer Benutzerfreundlichen Form darzustellen. Da die verschiedene Wahlsysteme jedoch nur nur unterschiedliche Visualisierungen Sinn machen ist es nötig eine Allgemeine Diagramm Schnittstelle zu bieten. Diese Schnittstelle ermöglicht es dem Diagramm die benötigten Daten entgegen zunehmen und durch die Java Swing Graphics Schnittstelle die in der Schablonenmethode drawChart(g : Graphics) überreicht wird zu zeichnen. Es ist jeder Implementation dieser Schnittstelle überlassen wie diese Daten zu interpretieren sind. Für die im Pflichtenheft beschrieben Wahlsysteme zu unterstützen, existieren die Implementationen \textit{PieChart} und \textit{StackedBarChart}.
		\subsubsection{Tabellen}
		Tabellen in verschiedenem Aussehen und Funktionalität werden in verschieden Bereichen der View benötigt (bsp. Wahlkonfiguration, Stimmabgabe und Ergebniseinsicht). Um redundanten UI Quellcode zu vermeiden wurde eine erweiterbare Tabelle entworfen, welche aus Verschiedenen Modulen (Erweiterungen) besteht mit denen sich alle bestehenden Anforderungen dieses Projektes erfüllen lassen.
		\\
		Grundlegend ist eine Liste eine Liste and Einträgen. Jeder Eintrag ist grundsätzlich gleich aufgebaut (besteht aus den gleichen Komponenten). Ein Eintrag ist durch die Klasse \textit{Entry} modelliert. Die Klasse \textit{BasicList} implementiert das grundlegende Verhalten einer Liste. Dieses erfasst die Möglichkeit durch die Tabelle zu scrollen, wenn sie nicht ganz in ihren Bereich passt und bietet die normale Schnittstelle einer jeden anderen Java Swing Komponente damit sie von dem Framework ohne Schwierigkeiten in die GUI integriert werden kann.
		\\
		Die Modularität dieses Tabellen Systems wird durch die Klasse \textit{ExtendableList} implementiert. Diese Klasse ist eine Erweiterung von \textit{BasicList}. Objekte die eine Instanz von \textit{ListExtension} sind können durch den Konstruktor von \textit{ExtendableList} zu der Tabelle hinzugefügt werden. Erweiterungen sind als Dekorierer Entwurfsmuster umgesetzt. Sie werden also zueinander gekettet indem jede Erweiterung eine andere durch ihren Konstruktor bekommen kann. Diese Kette an Erweiterungen wird schließlich dem Konstruktor der \textit{ExtendableList} überreicht der die von den Erweiterungen ausgehende Funktionalität umsetzen kann. Die Schnittstelle \textit{ListExtension} benachrichtigt jede Erweiterung wenn sich etwas in der Tabelle ändert (Ein Eintrag wird entfernt oder hinzugefügt). Wenn ein neuer Eintrag hinzugefügt (addEntry(e : Entry)) wird, wird ein leeres Objekt vom Typ \textit{Entry} durch die Kette der Erweiterungen gereicht und jeder Erweiterung die Möglichkeit geboten eigene Java Swing Komponenten auf diesem Eintrag zu platzieren. Sobald dieser Eintrag durch die komplette gewandert ist wird er in der Tabelle platziert. Es ist die Aufgabe jeder Erweiterung ihre Komponenten selber zu verwalten, da die eigentliche Tabelle diese nicht kennt. Da viele Tabellen Erweiterungen nur eine Komponente zu einem Eintrag hinzufügen ist diese Spezielle Funktionalität in der Klasse \textit{ComponentExtension} implementiert. Sie bietet außerdem eigene Schablonenmethoden über die Implementation starken Einfluss auf das konkrete Verhalten dieser Komponente nehmen können. Die Schablonenmethode newType() erlaubt der Implementation konkret zu bestimmen genaue welche Komponente zu dem nächsten Entry hinzugefügt werden soll. removeData() löscht die Komponente die sich auf dem Eintrag befindet welcher gerade gelöscht wird. Diese Methode erlaubt es einem Objekt von \textit{ComponentExtension} seinen Zustand aufzuräumen. Zuletzt bietet die \textit{ComponentExtension} Klasse die Hilfsmethode searchIndex(e : Entry). Diese Methode gibt den Index eines Eintrags in der Tabelle zurück und wird für das verwalten der zusätzlichen Daten in einer Erweiterung benötigt.
		\\
		Folgende Implementation der \textit{ListExtension} werden benötigt:
		\begin{itemize}
			\item TODO
		\end{itemize}
	
		\subsection(SupervisorView)
		TODO
		
		\subsection(VoterView)
		TODO
	
	\newpage
	
\end{document}