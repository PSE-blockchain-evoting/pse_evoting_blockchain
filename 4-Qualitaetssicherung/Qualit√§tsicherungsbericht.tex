\documentclass[parskip=full]{scrartcl}
\usepackage[T1]{fontenc}
\usepackage[utf8]{inputenc}
\usepackage[ngerman]{babel}
\usepackage{hyperref}
\hypersetup{
	pdftitle={PSE: Blockchain-basiertes E-Voting - Qualitätssicherung},%
	,%
}
\usepackage{graphicx}
\usepackage{csquotes}
\usepackage[nonumberlist]{glossaries}
\usepackage{enumitem}
\usepackage{xcolor}
\usepackage{svg}
\usepackage[section]{placeins}

\makeatletter
\AtBeginDocument{%
	\expandafter\renewcommand\expandafter\subsection\expandafter{%
		\expandafter\@fb@secFB\subsection
	}%
}
\makeatother
\makeatletter
\AtBeginDocument{%
	\expandafter\renewcommand\expandafter\subsubsection\expandafter{%
		\expandafter\@fb@secFB\subsubsection
	}%
}
\makeatother

\addto\extrasngerman{\def\figureautorefname{Abb.}}
\newcommand{\textitx}[1]{\mbox{\textit{#1}}}
\newcommand{\fakeparagraph}[1]{\textbf{#1}}
%\renewcommand{\includesvg}[1][1]{}


\title{
	PSE:Blockchain-basiertes E-Voting \\
	Qualitätssicherungsbericht
}
\author{Tim Fröhlich, Achim Kriso, Philipp Schaback, David Schuldes, Artem Vasilev\\ Phasenverantwortlicher: Achim Kriso}



\begin{document}
\clearpage
\maketitle
\pagenumbering{gobble}
\newpage

\tableofcontents
\newpage
\pagenumbering{arabic}

\section{Einleitung}

\section{Benutzte Werkzeuge}

\section{Unit-Tests} %Code coverage etc.

\section{Wahl- und Auszählungsfunktionalität}
\subsection{Wahlfunktionalität}
Die Stimmabgabe erfolgt durch Aufruf der vote()-Methode im VoterControlToModelIF.
Hierbei wurde getestet, dass die Stimme unverfälscht an das Netzwerk über das VoterSDKInterface weitergegeben wird und die Stimme bei erfolgreicher Abgabe in der jeweiligen VotingSystem-Klasse zur späteren Auszählung gespeichert wird.


\subsection{Auszählungsfunktionalität}
Die Auszählung der Stimmen erfolgt im Statemanagement-Paket in der abhängig vom ausgewählten Wahlsystem erzeugten VotingSystem-Klasse.
Die Auszählung erfolgt in Zwei Phasen: In der ersten Phase werden alle abgegebenen Stimmen mittels der determineResult()-Methode gruppiert.
In der zweiten Phase werden die gruppierten Ergebnisse mittels der determineWinner()-Methode gemäß dem festgelegten Wahlsystem ausgezählt und so der Gewinner ermittelt.\\
Es wurde das korrekte Verhalten der determineResults()- und determineWinner()-Methode getestet.

\section{GUI-Testplan}

\section{Gefundene Fehler}

\section{Portabilität}

\section{Performanz}
\subsection{Speicher}
\subsection{Laufzeit}

\end{document}