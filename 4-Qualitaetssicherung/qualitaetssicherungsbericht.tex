\documentclass[parskip=full]{scrartcl}
\usepackage[T1]{fontenc}
\usepackage[utf8]{inputenc}
\usepackage[ngerman]{babel}
\usepackage{hyperref}
\hypersetup{
	pdftitle={PSE: Blockchain-basiertes E-Voting - Qualitätssicherung},%
	,%
}
\usepackage{graphicx}
\usepackage{csquotes}
\usepackage[nonumberlist]{glossaries}
\usepackage{enumitem}
\usepackage{xcolor}
\usepackage{svg}
\usepackage[section]{placeins}

\makeatletter
\AtBeginDocument{%
	\expandafter\renewcommand\expandafter\subsection\expandafter{%
		\expandafter\@fb@secFB\subsection
	}%
}
\makeatother
\makeatletter
\AtBeginDocument{%
	\expandafter\renewcommand\expandafter\subsubsection\expandafter{%
		\expandafter\@fb@secFB\subsubsection
	}%
}
\makeatother

\addto\extrasngerman{\def\figureautorefname{Abb.}}
\newcommand{\textitx}[1]{\mbox{\textit{#1}}}
\newcommand{\fakeparagraph}[1]{\textbf{#1}}
%\renewcommand{\includesvg}[1][1]{}

\usepackage{qualitaetssicherungsbericht}

\title{
	PSE:Blockchain-basiertes E-Voting \\
	Qualitätssicherungsbericht
}
\author{Tim Fröhlich, Achim Kriso, Philipp Schaback, David Schuldes, Artem Vasilev\\ Phasenverantwortlicher: Achim Kriso}

\begin{document}
\clearpage
\maketitle
\pagenumbering{gobble}
\newpage

\tableofcontents
\newpage
\pagenumbering{arabic}

\section{Einleitung}
Dieses Dokument erfasst Ergebnisse der Qualitätssicherungsphase des Projektes "Blockchain-basiertes Evoting" im Rahmen des Moduls Praxis der Softwareentwicklung (PSE) am Lehrstuhl \enquote{Anwendungsorientierte formale Verifikation - Prof. Dr. Beckert} am Karlsruher Institut für Technologie (KIT).\\
Ziel der Software ist es, die Wahlsysteme absolute- und relative Mehrheitswahl sowie Instant-Runoff-Voting unter Garantie der unverfälschten Stimmabgabe für wahlberechtigte Wähler zu gewährleisten.

\section{Benutzte Werkzeuge}

\section{Unit-Tests} %Code coverage etc.

\section{Wahl- und Auszählungsfunktionalität}

\section{GUI-Testplan}
\test{Ablauf einer Wahl}{tst:setup}
\teststep{\begin{itemize}
		\item Das Blockchain-Netzwerk ist aufgesetzt.
		\item Deutsches locale ist definiert.
		\end{itemize}}
		{Das Programm ist gestartet.}
		{Das Authentifizierungsmenü des Wahlleiters erscheint}

\teststep{}
		{\begin{enumerate}
				\item Eingabe des Benutzername des Wahlleiters in das entsprechende Eingabefeld.
				\item Eingabe des Passwort des Wahlleiters in das entsprechende Eingabefeld.
				\item Die obere Schaltfläche "Bestätigen" drücken.
		\end{enumerate}}
		{Die GUI ändert sich auf die Startseite des Wahlleiters.}
		
\teststep{}
		{Drücken der Schaltflache \enquote{Neue Wahl erstellen}}
		{Es erscheint das Konfigurationsmenü.}

\teststep{}
		{\begin{enumerate}
				\item Eingabe des Namen \enquote{Vorstandswahl 2018} in das Eingabefeld \enquote{Name}.
				\item Auswahl des Wahlsystems \enquote{Relative Mehrheitswahl}.
				\item Eingabe der Beschreibung \enquote{Die Wahl des neuen Vorstands.} in das Eingabefeld \enquote{Beschreibung}.
				\item Drücken der Schaltflache \enquote{Weiter}.
		\end{enumerate}}
		{\begin{enumerate}
				\item Das Konfigurationsmenü wechselt zu dem Zeitraum-Tab.
				\item Der der Name der \enquote{Vorstandswahl 2018} wird oben im Titel angezeigt.
		\end{enumerate}}

\teststep{}
		{Drücken der \enquote{Sofort} Schaltflache.}
		{\begin{enumerate}
				\item Das derzeitige Datum wird in der oberen Datum-Eingabe angezeigt.
				\item Die derzeitige Uhrzeit wird in der oberen Uhrzeit-Eingabe angezeigt.
		\end{enumerate}}
	
\teststep{}
		{\begin{enumerate}
				\item Eingabe 31.August 2018 in die untere Datum-Eingabe.
				\item Eingabe 00:00 in die untere Uhrzeit-Eingabe.
				\item Auswahl der \enquote{Wähler Prozentsatz} Extrabedingung.
				\item Eingabe 98\% in die Prozent-Eingabe.
				\item Drücken der Schaltfläche \enquote{Weiter}.
		\end{enumerate}}
		{Das Konfigurationsmenü wechselt zu dem Kandidaten-Tab.}

\teststep{}
		{Drücke die Schaltfläche \enquote{-}}
		{Die Liste in dem Konfigurationsmenü ist komplett leer.}
		
\teststep{}
		{Drücke die Schaltfläche \enquote{Kandidat Hinzufügen} zwei mal.}
		{Zwei Zeilen sind in der Liste erschienen.}
	
\teststep{}
		{\begin{enumerate}
			\item Gebe \enquote{Wolfgang Rudolf} in das erste Eingabefeld in der Liste ein.
			\item Gebe \enquote{Sabine Scholl} in das zweite Eingabefeld in der Liste ein.
		\end{enumerate}}
		{}

\teststep{}
		{Drücke die Schaltfläche \enquote{Wähler} im Konfigurationsmenü.}
		{Das Konfigurationsmenü wechselt zum dem Wähler-Tab}

\teststep{}
		{\begin{enumerate}
				\item Drücke die Schaltfläche \enquote{Wähler hinzufügen} 3 mal.
				\item Gebe den Namen \enquote{Max Mustermann} in das erste Eingabefeld der Liste ein.
				\item Gebe den Namen \enquote{Anna Meier} in das zweite Eingabefeld der Liste ein.
				\item Gebe den Namen \enquote{Ulrich Müller} in das dritte Eingabefeld der Liste ein.
				\item Gebe den Namen \enquote{Erich Schmitt} in das vierte Eingabefeld der Liste ein.
				\item Drücke die Schaltfläche \enquote{-} in der 3 Zeile der Liste.
		\end{enumerate}}
		{Die dritte Zeile verschwindet und die vorher vierte Zeile wird zur neuen dritten Zeile.}

\teststep{}
		{Drücke die Schaltfläche \enquote{Weiter}}
		{\begin{enumerate}
				\item Es kommt eine Meldung, das die Wahlkonfiguration erfolgreich akzeptiert wurde.
				\item Es wird der Fertigstellen-Tab in dem Konfigurationsmenü angezeigt.
		\end{enumerate}}

\teststep{}
		{Drücke die \enquote{Bestätigen} Schaltfläche.}
		{\begin{enumerate}
				\item Eine Meldung teilt mit, dass die Wahl erfolgreich gestartet wurde.
				\item Es wird der aktuelle Wahlzustand in der GUI angezeigt.
				\item Links ist eine Tabelle in der jeder Kandidat eine Zeile hat in welcher der Name und Anzahl Stimmen angezeigt werden. Jede Zeile ist in einer Eigenen Farbe.
				\item In dem Verzeichnis indem das Programm ausgeführt wurde, befinden sich in Zertifikat für jeden Wähler und ein Zertifikat für den Wahlleiter.
		\end{enumerate}}

\teststep{}
		{Drücke die Schaltfläche \enquote{Allgemeine Infos}}
		{Unter der Schaltfläche erscheint ein Kasten in dem die Informationen die konfiguriert wurden angezeigt werden.}

\teststep{}
		{Drücke die Schaltfläche \enquote{Beenden}}
		{Die GUI schließt sich und das Programm terminiert.}

\teststep{}
		{Starten des Wähler-Clienten}
		{Die GUI des Wähler-Clienten wird angezeigt und der Benutzer dazu aufgefordert sein Zertifikat anzugeben.}

\teststep{}
		{Drücken der \enquote{Dursuchen} Schaltfläche.}
		{Ein Datei-Wahl-Menü erscheint.}
	
\teststep{}
		{\begin{enumerate}
				\item Auswählen des Wähler Zertifikats von "Max Musterman"
				\item \enquote{Öffnen} Schaltfläche drücken.
		\end{enumerate}}
		{Der Pfad des Zertifikats wird in dem Textfeld neben der "Durchsuchen" Schaltfläche angezeigt.}
		
\teststep{}
		{Drücken der \enquote{Bestätigen} Schaltfläche.}
		{Die GUI ändert sich und zeigt das Wahl-Menü an. Eine Tabelle links die eine Zeile für jeden Kandidaten enthält.}

\teststep{}
		{Drücken der \enquote{!} Schaltfläche in der ersten Zeile der Tabelle links.}
		{Es erscheint ein Beschreibungs-Fenster in dem die Beschreibung für \enquote{Wolgang Rudolf} steht.}

\teststep{}
		{Drücke die Schaltfläche \enquote{Allgemeine Infos}}
		{Unter der Schaltfläche erscheint ein Kasten in dem die Informationen die konfiguriert wurden angezeigt werden.}

\teststep{}
		{Klicke den Kreis links von \enquote{Sabine Scholl}}
		{Der Kreis ist ausgefüllt.}
	
\teststep{}
		{Drücke die Schaltfläche \enquote{Bestätigen}}
		{Es erscheint eine Meldung die nachfragt ob man sich seiner sicher Entscheidung sicher sei.}

\teststep{}
		{Drücke die Schaltfläche \enquote{Abbrechen} in der Meldung.}
		{Die Meldung verschwindet.}
			

\teststep{}
		{Drücke die Schaltfläche \enquote{Bestätigen}}
		{Es erscheint eine Meldung die nachfragt ob man sich seiner sicher Entscheidung sicher sei.}

\teststep{}
		{Drücke die Schaltfläche \enquote{OK}}
		{\begin{enumerate}
			\item Es erscheint eine Meldung das die Stimme erfolgreich abgegeben wurde.
			\item Die GUI zeigt nun an das man auf das Wahlende warten soll.
		\end{enumerate}}

\section{Gefundene Fehler}

\section{Portabilität}

\section{Performanz}
\subsection{Speicher}
\subsection{Laufzeit}

\end{document}
