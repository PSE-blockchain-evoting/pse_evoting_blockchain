\documentclass[parskip=full]{scrartcl}
\usepackage[T1]{fontenc}
\usepackage[utf8]{inputenc}
\usepackage[ngerman]{babel}
\usepackage{hyperref}
\hypersetup{
	pdftitle={PSE: Blockchain-basiertes E-Voting - Qualitätssicherung},%
	,%
}
\usepackage{graphicx}
\usepackage{csquotes}
\usepackage[nonumberlist]{glossaries}
\usepackage{enumitem}
\usepackage{xcolor}
\usepackage{svg}
\usepackage[section]{placeins}

\makeatletter
\AtBeginDocument{%
	\expandafter\renewcommand\expandafter\subsection\expandafter{%
		\expandafter\@fb@secFB\subsection
	}%
}
\makeatother
\makeatletter
\AtBeginDocument{%
	\expandafter\renewcommand\expandafter\subsubsection\expandafter{%
		\expandafter\@fb@secFB\subsubsection
	}%
}
\makeatother

\addto\extrasngerman{\def\figureautorefname{Abb.}}
\newcommand{\textitx}[1]{\mbox{\textit{#1}}}
\newcommand{\fakeparagraph}[1]{\textbf{#1}}
%\renewcommand{\includesvg}[1][1]{}

\usepackage{qualitaetssicherungsbericht}

\title{
	PSE:Blockchain-basiertes E-Voting \\
	Qualitätssicherungsbericht
}
\author{Tim Fröhlich, Achim Kriso, Philipp Schaback, David Schuldes, Artem Vasilev\\ Phasenverantwortlicher: Achim Kriso}

\begin{document}
\clearpage
\maketitle
\pagenumbering{gobble}
\newpage

\tableofcontents
\newpage
\pagenumbering{arabic}

\section{Einleitung}
Dieses Dokument erfasst Ergebnisse der Qualitätssicherungsphase des Projektes "Blockchain-basiertes Evoting" im Rahmen des Moduls Praxis der Softwareentwicklung (PSE) am Lehrstuhl \enquote{Anwendungsorientierte formale Verifikation - Prof. Dr. Beckert} am Karlsruher Institut für Technologie (KIT).\\
Ziel der Software ist es, die Wahlsysteme absolute- und relative Mehrheitswahl sowie Instant-Runoff-Voting unter Garantie der unverfälschten Stimmabgabe für wahlberechtigte Wähler zu gewährleisten.

\section{Benutzte Werkzeuge}
\subsection{Unit Tests}
Zum Durchführen der Unit Tests kommt JUnit in Verbindung mit Maven zum automatischen Ausführen der Tests zum Einsatz. 
\subsection{Codecoverage}
Die Testabdeckung wird über Codecov in Verbindung mit Travis CI ausgeführt. Bei Bedarf kann sie auch lokal mit JetBrains Intellij Coverage überprüft werden
\subsection{Bug Tracking}
Um den Überblick über gefundene Fehler und Verbesserungsmöglichkeiten zu behalten, benutzen wir GitHub's Issue Tracker

\section{Unit-Tests} %Code coverage etc.
\subsection{IRVVotingSystemTest-Klasse}  
Testen die Arbeit der Methoden loadVote(), determineWinner() und determineResults() der Klasse IRVVotingSystem. 

Um das Ziel zu implementieren, wurden die Stubs  der folgenden Schnittstellen / Klassen verwendet: 
\begin{enumerate}
	\item ElectionDataIF. 
	\item Election. 
\end{enumerate}

Der Test wurde in einem Testfall ähnlich dem im VoterElectionTest Test durchgeführt: 

\begin{enumerate}
	\item[1.] Fünf Stimmen wurden nacheinander hochgeladen: 
	\item[1.1.] \verb|["Bob", "Bill", "Sue"]| 
	\item[1.2.] \verb|["Sue", "Bob", "Bill"]| 
	\item[1.3.] \verb|["Bill", "Sue", "Bob"]| 
	\item[1.4.] \verb|["Bob", "Bill", "Sue"]|
	\item[1.5.] \verb|["Sue", "Bob", "Bill"]|

	Dazu wird loadVote() der IRVVotingSystem-Klasse verwendet. 
	\item[2.] Dann wurden die Methoden determineWinner() und determineResults() der gleichen Klasse mit der Überprüfung der Korrektheit des Ergebnisses mit Methoden der Familie assertX() durchgeführt. 
\end{enumerate}

Während des Tests wurden Fehler in der Klasse IRVVotingSystem erkannt und anschließend korrigiert.

\subsection{SupervisorControlTest} 
Methode getFirstAuthenticationListener() der Klasse SupervisorControl und Methode actionPerformed() der Klasse FirstAuthenticationListener zu überprüfen. 

Um das Ziel zu erreichen, wurden die Stubs der folgenden Schnittstellen / Klassen verwendet: 
\begin{enumerate}
	\item SupervisorControlToViewIF (Methoden getUsername() und getPassword ()). 
	\item SupervisorControlToModelIF. 
\end{enumerate}

Die Klasse testet, ob die Methode GetFirstAuthenticationListener() der Klasse SupervisorControl eine solche ActionListener zurückgibt, wenn die actionPerformed()  ausgeführt wird, die showFrontpage () der SupervisorControlToViewIF genau einmal aufgerufen wird und die firstAuthentication() der Schnittstelle SupervisorControlToModelIF mit den Argumenten login und password, die als Stub für die Schnittstelle SupervisorControlToViewIF eingegeben wurden. 

Diese Überprüfung wird mit den Methoden verify() und times() der Klassen Mockito und VerificationMode implementiert, zum Beispiel: 
verifizieren(view, times(1)).showFrontpage();

\section{Wahl- und Auszählungsfunktionalität}
\subsection{Wahlfunktionalität}
Die Stimmabgabe erfolgt durch Aufruf der vote()-Methode im VoterControlToModelIF.
Hierbei wurde getestet, dass die Stimme unverfälscht an das Netzwerk über das VoterSDKInterface weitergegeben wird und die Stimme bei erfolgreicher Abgabe in der jeweiligen VotingSystem-Klasse zur späteren Auszählung gespeichert wird.


\subsection{Auszählungsfunktionalität}
Die Auszählung der Stimmen erfolgt im Statemanagement-Paket in der abhängig vom ausgewählten Wahlsystem erzeugten VotingSystem-Klasse.
Die Auszählung erfolgt in Zwei Phasen: In der ersten Phase werden alle abgegebenen Stimmen mittels der determineResult()-Methode gruppiert.
In der zweiten Phase werden die gruppierten Ergebnisse mittels der determineWinner()-Methode gemäß dem festgelegten Wahlsystem ausgezählt und so der Gewinner ermittelt.\\
Es wurde das korrekte Verhalten der determineResults()- und determineWinner()-Methode getestet.

\section{GUI-Testplan}
\test{Ablauf einer Wahl}{tst:setup}
\teststep{\begin{itemize}
		\item Das Blockchain-Netzwerk ist aufgesetzt.
		\item Deutsches locale ist definiert.
		\end{itemize}}
		{Das Programm ist gestartet.}
		{Das Authentifizierungsmenü des Wahlleiters erscheint}

\teststep{}
		{Drücken der oberen \enquote{Bestätigen}-Schaltfläche.}
		{Es erscheint eine Fehlermeldung, die darauf hinweist, dass der Benutzername oder Passwort falsch ist.}

\teststep{}
		{}
		{Drücken der unteren \enquote{Bestätigen}-Schaltfläche.}
		{Es erscheint eine Fehlermeldung die darauf hinweist, dass das Zertifikat ungültig ist.}

\teststep{}
		{\begin{enumerate}
				\item Eingabe des Benutzername des Wahlleiters in das entsprechende Eingabefeld.
				\item Eingabe des Passworts des Wahlleiters in das entsprechende Eingabefeld.
				\item Die obere Schaltfläche \enquote{Bestätigen} drücken.
		\end{enumerate}}
		{Die GUI ändert sich auf die Startseite des Wahlleiters.}
		
\teststep{}
		{Drücken der Schaltflache \enquote{Neue Wahl erstellen}.}
		{Es erscheint das Konfigurationsmenü.}

\teststep{}
		{Drücken der Schaltfläche \enquote{Fertigstellen}.}
		{Der Fertigstellen-Tab des Konfigurationsmenü wird angezeigt mit folgenden Fehlern:
		\begin{enumerate}
			\item Der Wahlname muss aus mindestem einem Wort bestehen.
			\item Ein End- und Start-Zeitpunkt muss definiert sein.
			\item Es muss mindestens Zwei Kandidaten geben.
			\item Es muss mindestens Zwei Wähler geben.
		\end{enumerate}}

\teststep{}
		{\begin{enumerate}
				\item Drücken der Schaltfläche \enquote{Allgemein}.
				\item Eingabe des Namens \enquote{Vorstandswahl 2018} in das Eingabefeld \enquote{Name}.
				\item Auswahl des Wahlsystems \enquote{Relative Mehrheitswahl}.
				\item Eingabe der Beschreibung \enquote{Die Wahl des neuen Vorstands.} in das Eingabefeld \enquote{Beschreibung}.
				\item Drücken der Schaltflache \enquote{Weiter}.
		\end{enumerate}}
		{\begin{enumerate}
				\item Das Konfigurationsmenü wechselt zu dem Zeitraum-Tab.
				\item Der der Name der \enquote{Vorstandswahl 2018} wird als Titel der Wahl angezeigt.
		\end{enumerate}}

\teststep{}
		{Drücken der \enquote{Sofort}-Schaltflache.}
		{\begin{enumerate}
				\item Das derzeitige Datum wird in der oberen Datum-Eingabe angezeigt.
				\item Die derzeitige Uhrzeit wird in der oberen Uhrzeit-Eingabe angezeigt.
		\end{enumerate}}
	
\teststep{}
		{\begin{enumerate}
				\item Eingabe 31.August 2018 in die untere Datum-Eingabe.
				\item Eingabe 00:00 in die untere Uhrzeit-Eingabe.
				\item Auswahl der \enquote{Wähler Prozentsatz}-Extrabedingung.
				\item Eingabe 98\% in die Prozent-Eingabe.
				\item Drücken der Schaltfläche \enquote{Weiter}.
		\end{enumerate}}
		{Das Konfigurationsmenü wechselt zu dem Kandidaten-Tab.}

\teststep{}
		{Drücke die Schaltfläche \enquote{-}.}
		{Die Liste in dem Konfigurationsmenü ist komplett leer.}
		
\teststep{}
		{Drücke die Schaltfläche \enquote{Kandidat Hinzufügen} zweimal.}
		{Zwei Zeilen sind in der Liste erschienen.}
	
\teststep{}
		{\begin{enumerate}
			\item Gebe \enquote{Wolfgang Rudolf} in das erste Eingabefeld in der Liste ein.
			\item Gebe \enquote{Sabine Scholl} in das zweite Eingabefeld in der Liste ein.
		\end{enumerate}}
		{}

\teststep{}
		{Drücke die Schaltfläche \enquote{Wähler} im Konfigurationsmenü.}
		{Das Konfigurationsmenü wechselt zum dem Wähler-Tab}

\teststep{}
		{\begin{enumerate}
				\item Drücke die Schaltfläche \enquote{Wähler hinzufügen} dreimal.
				\item Gebe den Namen \enquote{Max Mustermann} in das erste Eingabefeld der Liste ein.
				\item Gebe den Namen \enquote{Anna Meier} in das zweite Eingabefeld der Liste ein.
				\item Gebe den Namen \enquote{Ulrich Müller} in das dritte Eingabefeld der Liste ein.
				\item Gebe den Namen \enquote{Erich Schmitt} in das vierte Eingabefeld der Liste ein.
				\item Drücke die Schaltfläche \enquote{-} in der dritten Zeile der Liste.
		\end{enumerate}}
		{Die dritte Zeile verschwindet und die vorherige vierte Zeile wird zur neuen dritten Zeile.}

\teststep{}
		{Drücke die Schaltfläche \enquote{Weiter}.}
		{\begin{enumerate}
				\item Es kommt eine Meldung, dass die Wahlkonfiguration erfolgreich akzeptiert wurde.
				\item Es wird der Fertigstellen-Tab in dem Konfigurationsmenü angezeigt.
		\end{enumerate}}

\teststep{}
		{Drücke die Schaltfläche \enquote{Exportieren}.}
		{Es erscheint ein Dateiauswahlmenü}

\teststep{}
		{Wähle einen Speicherplatz für die zu exportierende Wahl und drücke \enquote{OK}.}
		{Eine Datei ist an diesem Speicherplatz erschienen.}

\teststep{}
		{Drücke die Schaltfläche \enquote{Abbrechen}.}
		{Das Konfigurationsmenü schließt sich und der prozess terminiert.}

\teststep{}
		{Terminiere das Programm und starte es erneuet. Wiederhole die Anmeldung.}
		{Die Startseite der GUI wird angezeigt.}

\teststep{}
		{Drücke die \enquote{Wahl importieren} Schaltfläche.}
		{Es erscheint ein Dateiauswahlmenü}

\teststep{}
		{Wähle die Datei aus die an dem vorher gewählten Speicherplatz erschienen ist und drücke die Schaltfläche \enquote{Öffnen}.}
		{Es erscheint ein Konfigurationsmenü, in dem alle vorher eingetragen Daten enthalten sind.}

\teststep{}
		{Drücke die \enquote{Bestätigen} Schaltfläche.}
		{\begin{enumerate}
				\item Eine Meldung teilt mit, dass die Wahl erfolgreich gestartet wurde.
				\item Es wird der aktuelle Wahlzustand in der GUI angezeigt.
				\item Links ist eine Tabelle in der jeder Kandidat eine Zeile hat, in welcher der Name und Anzahl Stimmen angezeigt werden. Jede Zeile hat eine eigene Färbung.
				\item In dem Ordner-Verzeichnis, wo das Programm ausgeführt wurde, befindet sich ein Zertifikat für jeden Wähler und ein Zertifikat für den Wahlleiter.
		\end{enumerate}}

\teststep{}
		{Drücke die Schaltfläche \enquote{Allgemeine Infos}.}
		{Unter der Schaltfläche erscheint ein Kasten, in dem die Informationen welche konfiguriert wurden angezeigt werden.}

\teststep{}
		{Drücke die Schaltfläche \enquote{Beenden}.}
		{Die GUI schließt sich und das Programm terminiert.}

\teststep{}
		{Starten des Wähler-Clienten.}
		{Die GUI des Wähler-Clienten wird angezeigt und der Benutzer dazu aufgefordert sein Zertifikat anzugeben.}

\teststep{}
		{Drücken der \enquote{Dursuchen}-Schaltfläche.}
		{Ein Datei-Wahl-Menü erscheint.}
	
\teststep{}
		{\begin{enumerate}
				\item Auswählen des Wähler-Zertifikats von "Max Musterman".
				\item \enquote{Öffnen}-Schaltfläche drücken.
		\end{enumerate}}
		{Der Pfad des Zertifikats wird in dem Textfeld neben der "Durchsuchen" Schaltfläche angezeigt.}
		
\teststep{}
		{Drücken der \enquote{Bestätigen}-Schaltfläche.}
		{Die GUI ändert sich und zeigt das Wahl-Menü in Form einer Tabelle auf der linken Bildschirmseite, welche einen Zeileneintrag für jeden Kandidaten enthält.}

\teststep{}
		{Drücken der \enquote{!}-Schaltfläche in der ersten Zeile der Wahl-Menü-Tabelle.}
		{Es erscheint ein Beschreibungsfenster, in dem die Beschreibung für \enquote{Wolgang Rudolf} steht.}

\teststep{}
		{Drücke die Schaltfläche \enquote{Allgemeine Infos}.}
		{Unter der Schaltfläche erscheint ein Kasten, in dem die vorher festgelegten Informationen angezeigt werden.}

\teststep{}
		{Klicke den Kreis links von \enquote{Sabine Scholl}.}
		{Der Kreis ist ausgefüllt.}
	
\teststep{}
		{Drücke die Schaltfläche \enquote{Bestätigen}.}
		{Es erscheint eine Meldung, die nachfragt, ob man sich seiner sicher Entscheidung sicher sei.}

\teststep{}
		{Drücke die Schaltfläche \enquote{Abbrechen} in der Meldung.}
		{Die Meldung verschwindet.}
			

\teststep{}
		{Drücke die Schaltfläche \enquote{Bestätigen}.}
		{Es erscheint eine Meldung, die nachfragt, ob man sich seiner sicher Entscheidung sicher sei.}

\teststep{}
		{Drücke die Schaltfläche \enquote{OK}.}
		{\begin{enumerate}
			\item Es erscheint eine Meldung, dass die Stimme erfolgreich abgegeben wurde.
			\item Die GUI zeigt nun an, dass man auf das Wahlende warten soll.
		\end{enumerate}}

\teststep{}
		{Drücke die Schaltfläche \enquote{Beenden}.}
		{Die GUI des Wählers schließt sich und das Programm terminiert.}

\teststep{}
		{Starte den Klienten des Wahlleiters.}
		{Das Authentifizierungsmenü des Wahlleiters erscheint.}

\teststep{}
		{Drücke die \enquote{Durchsuchen}-Schaltfläche.}
		{Es erscheint ein Dateiauswahlmenü.}

\teststep{}
		{Wähle das Zertifikat des Wahlleiters im working-directory des Programmes aus und drücke die untere Schaltfläche \enquote{Bestätigen}.}
		{Die GUI zeigt den Zustand der Wahl an. Sabine Scholl hat eine Stimme und rechts wird ein Kreis mit gleicher Färbung wie der Eintrag von Sabine Scholl angezeigt.}
	
\teststep{}
		{Drücke die Schaltfläche \enquote{Wahl endgültig beenden}.}
		{Es erscheint eine Meldung, die darauf hinweist, dass dadurch die Wahl endgültig beendet wird und dieser Schritt nicht rückgängig gemacht werden kann, die Wahl neu gestartet und das Programm beendet wird.}
		
\teststep{}
		{Drücke die Schaltfläche \enquote{OK}.}
		{Die GUI schließt sich.}

\section{Gefundene Probleme und deren Lösung}
\subsection{Umstrukturierung zu polling-basierter Kommunikation.}
Wie im Entwurfsdokument beschrieben, sollte über die von Hyperledger-Fabric zur Verfügung gestellten Events die Kommunikation mit dem Blockchain-Netzwerk erfolgen, insbesondere um zu erfragen, ob die Wahl beendet ist. Da ausgelöste Events jedoch nicht vom Hyperledger-Fabric-SDK empfangen werden, ist eine solche Implementierung nicht möglich. Daher wurde beschlossen, ein polling-Verfahren zu verwenden. Dabei fragt ein Klient in regelmäßigen Zeitabständen das Blockchain-Netzwerk über Chaincdoe an, ob die Wahl geendet hat.
\\
Die dafür benötigten Änderungen beschränken sich auf das Model-Paket. Bei der Event-basierten Architektur gab es einen Listener, dessen Interface von der SDKConnection definiert und vom Statemanagement implementiert wurde. Diese Listener-Funktionalität wurde komplett entfernt. Das hat zur Folge, dass die Beziehung zwischen der SDKConnection und dem Statemanagement unidirektional ist. Außerdem wurde der Thread, welcher zuvor regelmäßig neue Events in dem Blockchain-Netzwerk ausgelöst hatte, verändert, so dass er in regelmäßigen Zeitabständen ermittelt, ob die Wahl geendet hat.
\\
Insgesamt wurden für diese Änderung 33 Zeilen hinzugefügt und 36 Zeilen gelöscht.

\section{Portabilität}

\section{Performanz}
\subsection{Speicher}
\subsection{Laufzeit}

\end{document}
