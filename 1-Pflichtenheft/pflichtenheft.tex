\documentclass[parskip=full,11pt,twoside]{scrartcl}
\usepackage[utf8]{inputenc}

\title{Blockchain-basiertes E-Voting}
\author{Artem Vasilev, Achim Kriso, Philipp Schaback, Tim Fröhlich, David Schuldes}

% section numbers in margins:
\renewcommand\sectionlinesformat[4]{\makebox[0pt][r]{#3}#4}

% header & footer
\usepackage{scrlayer-scrpage}
\lofoot{\today}
\refoot{\today}
\pagestyle{scrheadings}

\usepackage[sfdefault,light]{roboto}
\usepackage[T1]{fontenc}
\usepackage[german]{babel}
\usepackage[yyyymmdd]{datetime} % must be after babel
\renewcommand{\dateseparator}{-} % ISO8601 date format
\usepackage{hyperref}
\usepackage{amsmath} % for $\text{}$
\usepackage[nameinlink]{cleveref}
\crefname{figure}{Abb}{Abb}
\usepackage[section]{placeins}
\usepackage{xcolor}
\usepackage{graphicx}
\hypersetup{
	pdftitle={Pflichtenheft},
	bookmarks=true,
}
\usepackage{csquotes}
\usepackage{float}

\usepackage{pflichtenheft}
\usepackage[nopostdot]{glossaries}
\makeglossaries
\newglossaryentry{Benutzer}{name={Benutzer},description={
	Person welche mit der Software interagiert
}}

\newglossaryentry{Wahl}{name={Wahl},description={
	Eine Sammlung von Stimmen für die Bestimmung einer oder mehrerer Personen, Parteien o.Ä., die bestimmte Ämter oder 
		Funktionen bekleiden sollen.
}}

\newglossaryentry{Waehler}{name={Wähler},description={
	Person die bei einer Wahl einen bestimmten Kandidaten o.Ä. wählt und die hierfür nötigen Berechtigungen hat.
}}

\newglossaryentry{Kandidat}{name={Kandidat},description={
	TODO
}}

\newglossaryentry{Stimme}{name={Stimme},description={
	TODO
}}

\newglossaryentry{Auszaehlungsverfahren}{name={Auszählungsverfahren},description={
	Die Methode um zu bestimmen, welcher Kandidat bei einer Wahl gewonnen hat.
}}

\newglossaryentry{Absolute-Mehrheitswahl}{name={Absolute Mehrheitswahl},description={
Auszählungsverfahren, bei dem der Kandidat gewinnt, der über 50\% der Stimmen hat. Ansonsten hat niemand gewonnen.
}}

\newglossaryentry{Relative-Mehrheitswahl}{name={Relative Mehrheitswahl},description={
Auszählungsverfahren, bei dem der Kandidat mit den meisten Stimmen gewinnt.
}}

\newglossaryentry{Instant-Runoff-Voting}{name={Instant-Runoff-Voting},description={
Auszählungsverfahren, bei dem der Wähler die Kandidaten nach Präferenz ordnet.
\begin{enumerate}
	\item Der Wähler setzt Kandidat auf Platz 1, einen anderen Kandidat auf Platz 2 und so weiter. Es können auch Kandidaten nicht platziert werden.
	\item Bei der Auszählung wird bestimmt welcher Kandidat die wenigsten ersten Stimmen bekommen hat. Dieser Kandidat wird dann aus allen Stimmen entfernt. Nachgeordnete Kandidaten rücken im gegebenen Fall auf.
	\item Schritt 2 wird so lange wiederholt bis nur noch 2 Kandidaten übrig sind. Der Kandidat mit den meisten ersten Stimmen hat gewonnen.
\end{enumerate}
}}

\newglossaryentry{Benutzeroberflaeche}{name={Benutzeroberfläche},description={
Grafische Benutzerschnittstelle zu einem Computer, welche eine leichte und intuitiver Benutzung der Software ermöglicht.
}}

\newglossaryentry{Zertifikat}{name={Zertifikat}, plural=Zertifikate,description={
Digitaler Datensatz, der Personen oder Objekten bestätigt und deren Authentizität und Integrität durch kryptografische Verfahren prüft.
}}

\newglossaryentry{Blockchain}{name={Blockchain},description={
Datenbanktechnologie bei der die einzelnen Datensegmente verknüpft werden um so Eigenschaften wie Unveränderbarkeit und Dezentralisierung zu erreichen.
}}

\newglossaryentry{Ledger}{name={Blockchain-Ledger},description={
Datenbank in einer Blockchain, welche alle Transaktionen, die auf der Blockchain ausgeführt wurden, enthält. In dem Kontext unserer Wahl sind die Transaktionen einzelne Stimmen. Jeder Peer in dem Netzwerk besitzt einen Ledger.
}}

\newglossaryentry{Netzwerk}{name={Blockchain-Netzwerk},description={
Menge an verknüpften \glslink{Peer}{Peers}, die zusammen einen synchronierten \gls{Ledger} verwalten.
}}

\newglossaryentry{Peer}{name={Peer},description={
Computer der eine Kopie des Ledgers enthält, diese mit anderen Peers synchronisiert und eine Schnittstelle zwischen Chaincode und Software bereitstellt.
}}

\newglossaryentry{Chaincode}{name={Chaincode},description={
Programme die auf der Blockchain laufen und als Schnittstelle eines Peers zur Blockchain dienen.
}}

\newglossaryentry{Logdatei}{name={Logdatei},description={
Datei welche alle Ereignisse (Informationen, Warnungen und Fehler) in einem System protokolliert.
}}

\newglossaryentry{Linux}{name={Linux},description={
Beliebtes Unix-like Betriebssystem.
}}

\newglossaryentry{Double-Spending-Problem}{name={Double-Spending-Problem},description={
Potentieller Fehler in digitalen Geldsystemen der es ermöglicht dieselbe digitale Geldeinheit mehrmals auszugeben. In Wahlsystemen entspricht das dem mehrfachen wählen.
}}

\newglossaryentry{Wahlleiter}{name={Wahlleiter},description={
Administrator der Wahl, legt die Einstellungen, Wähler und Kandidaten der Wahl fest.
}}

\newglossaryentry{Konfiguration}{name={Konfiguration},description={
Feste Belegung der einstellbaren Eigenschaften eines Systems (hier i.d.R die Wahl).
}}

\newglossaryentry{Benutzerdaten}{name={Benutzerdaten},description={
Daten über den Benutzer. (z.B Name, Alter, Geburtsdatum und Gewohnheiten)
}}

\newglossaryentry{Wahlberechtigung}{name={Wahlberechtigt},description={
Vom Wahlleiter als Wähler zugelassen. (BESPRECHEN)
}}

\newglossaryentry{Stimmabgabe}{name={Stimmabgabe},description={
Sobald ein Wähler in der GUI seine Stimme gewählt hat bestätigt er diese. Seine Stimme wird dann von der GUI an das Blockchain-Netzwerk gesendet. Sofern die Stimme gültig ist wird sie in den Ledger eingetragen. Ansonsten wird die Stimme verworfen. Eine Stimme ist genau dann gültig wenn der Wähler sich als Wahlberechtigt authentifizieren konnte.
}}

\newglossaryentry{Stimmenauszaehlung}{name={Stimmenauszählung},description={
Das Auslesen der Stimmen aus der Blockchain um zu bestimmen wer die Wahl gewonnen hat. Dabei wird beachtet welches Wahlverfahren für die Wahl definiert wurde.
}}

\newglossaryentry{Wahl-Ende}{name={Wahl-Ende},description={
Sobald die in der Wahl Konfiguration eingestellte Endbedingung erfüllt ist, beendet sich die Wahl. Es ist nicht mehr möglich gültige Stimmen abzugeben. Die Ergebnisse der Wahl werden für jeden Wähler auf dessen GUI ausgewertet und angezeigt.
}}

\newglossaryentry{Klient}{name={Klient},description={
Das Programm das auf den Computern der Wählern und des Wahlleiter läuft. Es stellt die GUI bereit und verwaltet die Kommunikation mit dem \gls{Netzwerk}.
}}

\newglossaryentry{Wahlstand}{name={Wahlstand},description={
		Das Auswertungsergebnis der Wahl, bei dem alle bisher erfolgreich abgegebenen Stimmen ausgezählt werden.
}}





\begin{document}
\maketitle

\pagebreak

\tableofcontents
\pagebreak
%%%%%%%%%%%%%%
\section{Produktübersicht}

Bei herkömmlichen E-Voting Lösungen gestaltet es sich als problematisch, Manipulation von Wahlergebnissen zu verhindern und den Wählern zu gewährleisten, dass ihre Stimme unverändert in die Wahl eingegangen ist. \\
Die Blockchain E-Voting Software löst diese Probleme mithilfe der Blockchain-Technologie. \\
Sobald ein Wähler seine Stimme abgibt wird sie im \gls{Ledger} gespeichert und kann nicht mehr verändert werden. Über diesen Weg garantiert die Software, dass seine Stimme nicht verloren geht und unverändert gespeichert wurde.

\section{Zielbestimmung}

\subsection{Musskriterien}

\criterium{Unterstützte Wahlformen}{crt:vote-formats}
Die Software unterstützt das Mehrheitswahl-Prinzip. Es ist möglich eine Wahl aufzusetzen, bei der der Gewinner nach dem Prinzip der 
\glslink{Relative-Mehrheitswahl}{Relativen Mehrheitswahl} ermittelt wird. 
Es wird sichergestellt, das der Wähler nur eine Stimme \glslink{Stimmabgabe}{abgeben} kann.

\criterium{Erstellung einer Wahl}{crt:create}
Ein \gls{Wahlleiter} kann eine \gls{Wahl} über seine \gls{Benutzeroberflaeche} \glslink{Konfiguration}{konfigurieren}.
Wenn eine \gls{Wahl} gestartet ist können nur \gls{Waehler} die als \glslink{Wahlberechtigung}{wahlberechtigt} authentifiziert sind die \gls{Wahl} in ihrer \gls{Benutzeroberflaeche} sehen.

\criterium{Teilnahme an einer Wahl}{crt:vote} 
Wenn ein \gls{Waehler} \glslink{Wahlberechtigung}{wahlberechtigt} ist hat er die Möglichkeit, abhängig von dem \gls{Auszaehlungsverfahren}, einen oder mehrere Kandidaten in seiner \gls{Benutzeroberflaeche} auszuwählen. Wenn beispielsweise das \gls{Instant-Runoff-Voting} ausgewählt ist kann der \gls{Waehler} mehrere \glslink{Kandidat}{Kandidaten} bewerten, während bei den Mehrheitswahlen nur ein \gls{Kandidat} ausgewählt werden kann.
Wenn der \gls{Waehler} seine Auswahl bestätigt wird sie vom \gls{Klient} an das Netzwerk übermittelt.

\criterium{Rückmeldung bei Stimmabgabe}{crt:vote-feedback}
Wenn ein \gls{Waehler} seine \gls{Stimme} abgegeben hat wird er über seine \gls{Benutzeroberflaeche} informiert, ob seine \gls{Stimmabgabe} erfolgreich war oder nicht. Eine \gls{Stimmabgabe} ist genau dann erfolgreich, wenn die \gls{Stimme} im \gls{Ledger} eingetragen wurde.

\criterium{Stimmenübermittlung}{crt:send}
Ist die \gls{Stimmabgabe} eines \glslink{Waehler}{Wählers} erfolgreich garantiert die Software, dass seine \gls{Stimme} unverändert zum \gls{Ledger} hinzugefügt wurde.

\criterium{Stimmenauszählung}{crt:count}
Sobald eine \gls{Wahl} \glslink{Wahl-Ende}{beendet} ist werden alle erfolgreich \glslink{Stimmabgabe}{abgegebenen} Stimmen mit einem \gls{Auszaehlungsverfahren} ausgezählt. Das \gls{Auszaehlungsverfahren} wird bei der \glslink{Konfiguration}{Wahlkonfiguration} festgelegt.

\criterium{Graphische Benutzeroberfläche für Wahlleiter}{crt:gui-es}
Der \gls{Wahlleiter} kann über seine \gls{Benutzeroberflaeche} eine Wahl \glslink{Konfiguration}{konfigurieren}. Ist eine \gls{Wahl} gestartet wird der aktuelle \gls{Wahlstand} auf seiner \gls{Benutzeroberflaeche} angezeigt. Ist die Wahl \glslink{Wahl-Ende}{beendet} wird das Ergebnis der \glslink{Stimmenauszaehlung}{Auszählung} auf der \gls{Benutzeroberflaeche} angezeigt.

\criterium{Graphische Benutzeroberfläche für Wähler}{crt:gui-voter}
Die \gls{Benutzeroberflaeche} des \glslink{Waehler}{Wählers} bietet Informationen über die \glslink{Kandidat}{Kandidaten}, erlaubt es für einen \glslink{Kandidat}{Kandidaten} zu stimmen. Ihm werden außerdem die Beschreibung der Wahl, als auch die Beschreibungen der Kandidaten angezeigt.

\subsection{Sollkriterien}

\criteriumShould{Absolute Mehrheitswahl}{crt:absfptp}
Wahlen können mit dem \gls{Auszaehlungsverfahren} der \glslink{Absolute-Mehrheitswahl}{Absoluten Mehrheitswahl} ausgewertet werden. Die \gls{Absolute-Mehrheitswahl} ist dementsprechend eine Option die dem \gls{Wahlleiter} zur Verfügung gestellt wird.
Sollte zum Ende der Wahl keiner der Kandidaten 50\% erreicht haben gibt es keinen Gewinner.
In diesem Fall wird dem Wahlleiter die Möglichkeit geboten die Wahl erneut zu starten.

\criteriumShould{Speichern der Konfiguration}{crt:safeconf}
Der \gls{Wahlleiter} kann eine \gls{Konfiguration} in einer Datei abspeichern und wieder laden. Diese Funktionalität ist in dem \gls{Konfigurationsmenu} erreichbar. Das Laden einer solchen Datei überschreibt alle Informationen die schon in dem \gls{Konfigurationsmenu} eingegeben wurden.

\criteriumShould{Dynamische Peer-Verbindung}{crt:dynamic-peer}
Beim Verbindungsaufbau eines \glslink{Klient}{Klienten} zum \gls{Netzwerk} verbindet sich dieser mit dem \gls{Peer} der die niedrigste \gls{Latenz} aufweist. Der \gls{Klient} ermittelt erst die \gls{Latenz} aller \glslink{Peer}{Peers} und verbindet sich dann mit demjenigen, der die niedrigste \gls{Latenz} aufweist.

\subsection{Kannkriterien}

\criteriumOptional{Instant-Runoff-Voting}{crt:irv}
Das \gls{Auszaehlungsverfahren} \gls{Instant-Runoff-Voting} kann benutzt werden um eine Wahl auszuwerten und wird dem \gls{Wahlleiter} zur Auswahl gestellt.

\criteriumOptional{Geheime Wahlen}{crt:secret}
Wähler können ihre Stimmen \glslink{Stimmabgabe}{abgeben}, so dass sie nur für den \gls{Wahlleiter} der Wahl einsehbar ist.
Das wird mit einem asymmetrischen Verschlüsselungsverfahren erreicht.
Das Wahlergebnis kann am Ende der Wahl vom Wahlleiter an die Wähler propagiert werden.

\criteriumOptional{Verteilung von Zugangsdaten}{crt:certdist}
Der \gls{Wahlleiter} kann die Zugangsdaten von \glslink{Waehler}{Wählern} (dessen \gls{Zertifikat}) für die \gls{Wahl} per Email an die \gls{Waehler} zu verteilen.

\criteriumOptional{Automatisches Wahlende}{crt:autoend}
Bestimmung einer weiteren Endbedingung bei der \glslink{Konfiguration}{Wahlkonfiguration} deren Erfüllung die \gls{Wahl} automatisch beendet.
Eine Wahl ist \glslink{Wahl-Ende}{beendet}, wenn der bei der \glslink{Konfiguration}{Wahlkonfiguration} festgelegte Endzeitpunkt der Wahl erreicht ist oder wenn die zusätzlich festgelegte Endbedingung erreicht ist.
Die zusätzlichen Endbedingungen sind:\\
\begin{itemize}
	\item Ein bei der \glslink{Konfiguration}{Wahlkonfiguration} festgelegter Prozentsatz an \glslink{Waehler}{Wählern} hat seine Stimme erfolgreich \gls{Stimmabgabe}{abgegeben}.\\
	\item Ein \gls{Kandidat} hat einen bei der \glslink{Konfiguration}{Wahlkonfiguration} festgelegten Prozentsatz an insgesamt zugelassenen \glslink{Waehler}{Wählern} erreicht.
\end{itemize}

\subsection{Abgrenzungskriterien}
\criteriumNot{Unveränderbarkeit einer Stimme}{cst:nochange}
Sobald der \gls{Waehler} eine Stimme \glslink{Stimmabgabe}{abgegeben} hat, kann er diese nicht mehr ändern. Das gilt insbesondere auch wenn die \gls{Wahl} noch läuft.

\criteriumNot{Kein Speichern von Wahlverhalten}{cst:nomem}
Der \gls{Ledger}, der die \glslink{Stimmabgabe}{Stimmabgaben} enthält wird nach finalem Beenden der \gls{Wahl} durch den \gls{Wahlleiter} gelöscht.
Deswegen ist das Wahlverhalten (die \glslink{Stimmabgabe}{Stimmabgaben} eines \glslink{Waehler}{Wählers} aus früheren \glslink{Wahl}{Wahlen}) für keinen Benutzer der Software einsehbar. 

\criteriumNot{Vertrauen in den Wahlleiter}{cst:trustes}
Die Legitimität einer \gls{Wahl} beruht auf der Vertrauenswürdigkeit des \glslink{Wahlleiter}{Wahlleiters}. Wenn der \gls{Wahlleiter} seine \gls{Wahl} manipulieren möchte, so sind keine Maßnahmen im System vorhanden die ihn davon stoppen können. Beispielsweise könnte der \gls{Wahlleiter} im Namen der \gls{Waehler} wählen, da er alle \glslink{Zertifikat}{Zertifikate} der \gls{Waehler} besitzt.

%%%%%%%%%%%
\section{Produkteinsatz}

\subsection{Anwendungsbereiche}
Die Software wird zur Durchführung von kleinen Wahlen oder Abstimmungen im Rahmen von Vereinen, Firmen oder Parlamenten verwendet.

\subsection{Zielgruppen}
\begin{enumerate}
  \item \gls{Waehler}
  \item \gls{Wahlleiter}
\end{enumerate}

\section{Produktumgebung}

\subsection{Software}
Die Software soll für das Erstellen und Verwalten der \gls{Blockchain} das Hyperledger Fabric Framework verwenden. Das Framework bietet eine auf dem \gls{Consensus-Verfahren} basierende Blockchain-Implementierung.
Hierfür wird die Hauptanwendung in Java 7 geschrieben und das Hyperledger Fabric Java SDK verwendet. Für das erstellen von \glslink{Chaincode}{Chaincodes} soll Go verwendet werden.
Um die Funktionalität von Hyperledger Fabric zu gewährleisten wird ebenso eine Installation von GoLang benötigt.

\subsection{Hardware}
Es werden Computer für die \gls{Waehler} und den \gls{Wahlleiter} benötigt. Diese Computer müssen mit einer Internetverbindung ausgestattet sein und mindestens Java 7 installiert haben.

\subsection{Orgware}
\begin{enumerate}
\item Installation der Software, die für das Funktionieren von Hyperledger erforderlich ist. (siehe http://hyperledger-fabric.readthedocs.io/en/release-1.1/prereqs.html)
\item Anlegen des Netzwerks, wenn es keines gibt.
\end{enumerate}

\subsection{Schnittstellen}
Die \gls{Wahl} eines \glslink{Waehler}{Wählers} wird im \gls{Ledger} gespeichert und über das \gls{Netzwerk} verteilt.

\section{Funktionale Anforderungen}

\functionality{Erstellung einer Wahl}{fnc:create}
\fulfills{crt:create}
\fulfills{crt:gui-es}
\fulfills{crt:safeconf}
Der \gls{Wahlleiter} kann eine neue \gls{Wahl} erstellen. Es existiert immer nur eine \gls{Wahl} zu einem Zeitpunkt.
Zudem kann eine Vorher abgespeicherte \glslink{Konfiguration}{Wahlkonfiguration} geladen werden. Hierbei öffnet sich das \gls{Konfigurationsmenu} mit den geladenen Einstellungen in der \hyperref[fig:wlltr-done]{Übersichtsseite}.
Der \gls{Wahlleiter} wird durch die Erstellung geführt. Es sind folgende Schritte notwendig:

\functionality{Allgemeine Konfiguration der Wahl}{fnc:params}
\fulfills{crt:create}
\fulfills{crt:gui-es}
\fulfills{crt:secret}
Der \gls{Wahlleiter} legt den Namen und einen Beschreibungstext der \gls{Wahl} fest.
Er setzt den Start- und Endzeitpunkt der \gls{Wahl}.

\functionality{Auswahl des Auszählungsverfahrens}{fnc:votesys}
\fulfills{crt:vote-formats}
\fulfills{crt:gui-es}
\fulfills{crt:absfptp}
\fulfills{crt:irv}
Die Festlegung eines \glslink{Auszaehlungsverfahren}{Auszählungsverfahrens} ist möglich. Es stehen die folgenden zur Verfügung:
\begin{itemize}
	\item \gls{Relative-Mehrheitswahl}
	\item \gls{Absolute-Mehrheitswahl}
	\item \gls{Instant-Runoff-Voting}
\end{itemize}

\functionality{Hinzufügen von Wählern}{fnc:voter-add}
\fulfills{crt:create}
\fulfills{crt:gui-es}
\fulfills{crt:certdist}
Der \gls{Wahlleiter} fügt die \gls{Waehler} mit ihrem Namen hinzu. Das System generiert automatisch die erforderlichen \glspl{Zertifikat}. Die \glspl{Zertifikat} werden im \glslink{Netzwerk}{Netzwerk} verteilt. Nur die hinzugefügten \gls{Waehler} sind zur Teilname berechtigt. \\
Der \gls{Wahlleiter} kann außerdem die \glspl{Zertifikat} per Email an die \gls{Waehler} senden.

\functionality{Importieren der Wahlkonfiguration}{fnc:import}
\fulfills{crt:create}
\fulfills{crt:gui-es}
\fulfills{crt:safeconf}
Der \gls{Wahlleiter} hat beim aufsetzen einer neuen \gls{Wahl} im \gls{Konfigurationsmenu} die Möglichkeit eine existierende Konfigurationsdatei zu benutzen um die \gls{Wahl} zu erstellen.
Die Konfigurationsdatei enthält Einstellungen um eine neue \gls{Wahl} aufzusetzen. Sie kann vom \gls{Wahlleiter} aus einer bereits aufgesetzten \gls{Wahl} \hyperref[fnc:export]{erzeugt} werden.

\functionality{Hinzufügen von Kandidaten}{fnc:candidate-add}
\fulfills{crt:create}
\fulfills{crt:gui-es}
Der \gls{Wahlleiter} fügt die \glslink{Kandidat}{Kandidaten} mit ihrem Namen hinzu. Es ist außerdem möglich jedem \glslink{Kandidat}{Kandidaten} eine Beschreibung zu geben, die den \glslink{Waehler}{Wählern} in ihrer \glslink{Benutzeroberflaeche}{GUI} angezeigt wird. Das System propagiert die \glslink{Kandidat}{Kandidaten} automatisch an das \glslink{Netzwerk}{Netzwerk}.

\functionality{Exportieren der Wahlkonfiguration}{fnc:export}
\fulfills{crt:create}
\fulfills{crt:gui-es}
\fulfills{crt:safeconf}
Der \gls{Wahlleiter} hat die Möglichkeit seine vorgenommenen Einstellungen in einer Datei zu exportieren.
Es wird ein \gls{Dateibrowser} geöffnet in welchem der Speicherort für die Konfigurationsdatei festgelegt werden kann.
Die gespeicherte Konfigurationsdatei beinhaltet alle Einstellungen der \gls{Wahl}.

\functionality{Aktivierung der Wahl}{fnc:activate}
\fulfills{crt:create}
\fulfills{crt:gui-es}
Der \gls{Wahlleiter} bestätigt seine Einstellungen zur \gls{Wahl}. Mit der Bestätigung beginnt die Übertragung der Informationen in das \glslink{Netzwerk}{Netzwerk}. Die \gls{Wahl} startet zum festgelegten Zeitpunkt.

\functionality{Wahlfunktion für Wähler}{fnc:vote}
\fulfills{crt:vote}
\fulfills{crt:send}
\fulfills{crt:gui-voter}
Der \gls{Waehler} kann an der laufenden \gls{Wahl} teilnehmen. Er durchläuft dazu folgende Schritte:

\functionality{Authentifizierung mittels Zertifikat}{fnc:auth}
\fulfills{crt:vote}
\fulfills{crt:gui-voter}
Der Wähler authentifiziert sich gegenüber dem \glslink{Netzwerk}{Netzwerk} mit seinem \gls{Zertifikat}.
Ist er zur \gls{Wahl} nicht berechtigt oder hat schon gewählt wird er darauf hingewiesen.
Andernfalls kann er wählen.

\functionality{Auswählen eines Kandidaten}{fnc:choose}
\fulfills{crt:vote}
\fulfills{crt:gui-voter}
Dem \gls{Waehler} stehen die vom \gls{Wahlleiter} festgelegten \glspl{Kandidat} zur Auswahl.
Der \gls{Waehler} kann einen der \glspl{Kandidat} auswählen.
Er kann seine Auswahl beliebig oft ändern.

\functionality{Übermittlung der Stimme}{fnc:send}
\fulfills{crt:vote}
\fulfills{crt:send}
Bestätigt er seine \gls{Wahl}, so wird die \gls{Stimme} in den \gls{Ledger} geschrieben.
Sie ist hiermit final übernommen und kann nicht länger geändert werden.

\functionality{Rückmeldung an den Wähler}{fnc:feedback}
\fulfills{crt:gui-voter}
\fulfills{crt:vote-feedback}
Wenn die \gls{Stimme} erfolgreich im \gls{Ledger} aufgenommen wurde, erhält der \gls{Waehler} eine Bestätigung über seine \glslink{Benutzeroberflaeche}{GUI}.
Sonst erhält der \gls{Waehler} eine Benachrichtigung, dass seine \gls{Wahl} fehlschlug.
Er kann dann zur Auswahl der \glspl{Kandidat} zurückkehren und seine \gls{Stimme} erneut abgeben.

\functionality{Beenden der Wahl}{fnc:end}
\fulfills{crt:count}
\fulfills{crt:autoend}
Die \gls{Wahl} endet zum eingestellten Zeitpunkt (oder bei Erreichen des zusätzlichen Kriteriums) automatisch. \glslink{Stimmabgabe}{Stimmabgaben} der \gls{Waehler} sind nicht länger möglich/gültig. Die Auswertung der \gls{Wahl} beginnt:

\functionality{Auszählung der Stimmen}{fnc:count}
\fulfills{crt:count}
Die \glslink{Stimmenauszaehlung}{Auszählung} findet in jedem \glslink{Klient}{Klienten} statt. Dieser fragt die \glspl{Stimme} von seinem \gls{Peer} ab und bestimmt, abhängig vom dem \gls{Auszaehlungsverfahren}, welcher \gls{Kandidat} gewonnen hat. Das Ergebnis der \glslink{Stimmenauszaehlung}{Auszählung} wird daraufhin auf den \glslink{Benutzeroberflaeche}{GUIs} des \glslink{Waehler}{Wählers} als auch des \glslink{Wahlleiter}{Wahlleiters} dargestellt.

\functionality{Dynamische Peer-Verbindung}{fnc:dynamic-peer}
\fulfills{crt:dynamic-peer}
Der \gls{Klient} verbindet sich automatisch mit einem \gls{Peer} im \gls{Netzwerk}. Dabei beachtet er die \glslink{Latenz}{Latenzen} aller \glslink{Peer}{Peers} und wählt den mit der kleinsten \gls{Latenz} aus.

\section{Produktdaten}

Die \glslink{Stimmabgabe}{Stimmabgaben} der Wahlteilnehmer werden auf einem \gls{Ledger} gespeichert.
Die \glspl{Zertifikat} der \gls{Waehler} werden in \glspl{Datenbank} im \glslink{Netzwerk}{Netzwerk} gespeichert.
Diese Daten gehen verloren sobald der \gls{Ledger} gelöscht wird. Es werden keine weiteren \gls{Benutzerdaten} gespeichert.

\section{Produktleistungen}
Die Software muss die abgegebene \gls{Wahl} eines jeden \glslink{Waehler}{Wählers} zählen. Eine abgegebene \gls{Stimme} ist unverfälscht im Wahlergebnis enthalten.
Jede abgegebene \gls{Stimme} muss in der \glslink{Stimmenauszaehlung}{Auszählung} der \gls{Wahl} vertreten und somit dargestellt sein. \\
Die Software muss dem \gls{Waehler} auf der \gls{Benutzeroberflaeche} deutlich machen ob seine \gls{Wahl} erfolgreich war oder nicht. \\
Die Abgabe einer \gls{Stimme} dauert nicht länger als 5 Minuten.
Die \glslink{Benutzeroberflaeche}{GUI} reagiert hierbei sofort um dem \gls{Benutzer} Rückmeldung über seine Aktion geben. \\
Die \glslink{Stimmenauszaehlung}{Auszählung} einer \gls{Wahl} unter einer Stimmenzahl von 10.000 \glslink{Waehler}{Wählern} dauert nicht länger als 5 Minuten.

\section{Weitere Nicht-Funktionale Anforderungen}

\nonFunctionality{Einfache Benutzung}{nfc:design}
Die \gls{Wahl} sollte für einen \gls{Benutzer} mit nur geringen Computerkenntnissen möglich sein.

\nonFunctionality{Verifizieren der Wahlberechtigung}{nfc:design}
Es soll nur denjenigen Benutzern möglich sein an einer \gls{Wahl} teilzunehmen, welche die hierzu notwendigen \glslink{Wahlberechtigung}{Berechtigungen} haben.

\nonFunctionality{Manipulation der Wahl}{nfc:design}
Es soll nicht möglich sein die \gls{Wahl} eines Anderen zu ändern, \glslink{Stimme}{Stimmen} zu löschen oder anderweitig das Ergebnis der \gls{Wahl} zu manipulieren.

\section{Qualitätsanforderungen}
\quality{Korrektheit der Wahlergebnisse}{qlt:correctness}
Sofern ein Wähler seine Stimme erfolgreich abgegeben hat, ist diese garantiert im Wahlergebnis enthalten. Sie wurde dem Kandidaten, für den gewählt wurde, angerechnet.
\quality{Protokollierung des Netzwerkes}{qlt:log}
Ereignisse und Probleme auf dem \gls{Netzwerk} werden in einer \gls{Logdatei} chronologisch protokolliert. Diese \gls{Logdatei} ist für den \gls{Wahlleiter} einsehbar.
\quality{Unveränderbarkeit der Wahl}{qlt:immutability}
Sobald die Wahl vom \gls{Wahlleiter} einmal aufgesetzt wurde, können die Einstellungen dieser Wahl von niemandem (insbesondere vom \gls{Wahlleiter}) mehr verändert werden.
\quality{Vermeidung von unlogischen Eigenschaften der Wahl}{qlt:illogical}
Während dem festlegen der Einstellungen der Wahl wird der \gls{Wahlleiter} auf Probleme in der \gls{Konfiguration} hingewiesen. Folgende Fehler werden berücksichtigt:
	\begin{itemize}
		\item Kein oder nur ein Kandidat wurde eingetragen.
		\item Kein oder nur ein Wähler wurde eingetragen.
		\item Einem Wähler oder Kandidaten wurde kein Name gegeben.
		\item Der Wahl wurde kein Name gegeben.
		\item Die Wahl endet vor oder demselben Zeitpunkt an dem sie startet.
	\end{itemize}
\quality{Verhinderung des \glslink{Double-Spending-Problem}{Double-Spending-Problems}}{qlt:ds}
Jeder Wähler kann nur einmal seine Stimme erfolgreich abgeben. Eine erneute Abgabe seiner Stimme resultiert in einer Fehlermeldung auf der Wähler GUI, die ihn auf das Problem hinweist.
\quality{Vermeidung von ungewollten Enthaltungen}{qlt:abstain}
Falls ein Wähler seine Wahl auf seiner GUI bestätigen möchte, ohne das dieser für einen der Kandidaten gestimmt hat, wird eine Warnung angezeigt die darauf hinweist.
\quality{Warnungen bei Netzwerkproblemen}{qlt:nw-problems}
Bei Problemen die Stimme eines Wählers in das \gls{Netzwerk} zu übertragen, wird dieser Wähler in seiner GUI darüber informiert. Er hat dann die Möglichkeit seine Stimme erneut abzugeben. 

\section{Globale Testfälle}
\test{Ablauf einer Wahl}{tst:wholeVote}
\tests{fnc:create}
\tests{fnc:params}
\tests{fnc:votesys}
\tests{fnc:voter-add}
\tests{fnc:candidate-add}
\tests{fnc:activate}
\tests{fnc:vote}
\tests{fnc:auth}
\tests{fnc:choose}
\tests{fnc:send}
\tests{fnc:feedback}
\tests{fnc:end}
\tests{fnc:count}
\tests{fnc:dynamic-peers}


\teststep{\gls{Wahlleiter} \enquote{Fritz Müller} hat die \glslink{Benutzeroberflaeche}{GUI} geöffnet um eine neue \gls{Wahl} aufzusetzen.}
		{Fritz gibt den Namen \enquote{Vorstandswahl 2018} an, wählt als Wahlsystem die \gls{Relative-Mehrheitswahl} aus und fügt eine passende Beschreibung hinzu.\\
		Fritz fügt \enquote{Max Mustermann}, \enquote{Anna Meier}, \enquote{Erich Schmitt} und 10 andere \gls{Waehler} hinzu.\\
		Zuletzt werden \enquote{Johannes Heinzhof}, \enquote{Wolfgang Rudolf} und \enquote{Sabine Scholl} als Wahlmöglichkeiten von Fritz festgelegt,}
		{Die \glspl{Zertifikat} für alle zugelassenen \gls{Waehler} werden generiert.}	
	
\teststep{}
		{Fritz wählt aus, dass die \gls{Wahl} am 1. August 2018 beginnt und am 31. August 2018 endet. Er bestätigt seine Eingaben.}
		{Die \gls{Wahl} ist jetzt im \gls{Netzwerk} aktiv.}

\teststep{Max Mustermann startet die \gls{Waehler} \glslink{Benutzeroberflaeche}{GUI} am 1. August 2018}
		{Max gibt seinen Namen und sein Authentifizierungs \gls{Zertifikat} an.}
		{Die \glslink{Benutzeroberflaeche}{GUI} updatet sich und zeigt alle Wahlmöglichkeiten an}
		
\teststep{}
		{Max wählt \enquote{Sabine Scholl} aus den Wahlmöglichkeiten aus und bestätigt seine Wahl.}
		{Die \glslink{Benutzeroberflaeche}{GUI} schickt seine \gls{Stimme} ab und informiert ihn dass diese erfolgreich gezählt wurde.}
		
\teststep{Der 31. August 2018 ist erreicht.}
		{Die \gls{Wahl} beendet sich auf dem \gls{Netzwerk}.}
		{Die Wahlergebnisse können auf den \glslink{Benutzeroberflaeche}{GUIs} eingesehen werden.}
		
\test{Verhinderung von Angriffsversuchen auf die Wahl}{tst:securityTest}
\tests{fnc:vote}
\tests{fnc:dynamic-peers}

\teststep{Wiederholung des Schritte von T1 1.1-1.4}
		{Max startet erneut eine \gls{Waehler} \glslink{Benutzeroberflaeche}{GUI} und gibt seine Namen und \gls{Zertifikat} an.}
		{Die \glslink{Benutzeroberflaeche}{GUI} teilt ihm mit, dass seine \gls{Stimme} schon gegeben wurde und bricht den Wahlvorgang ab.}

\teststep{}
		{Max startet erneut eine \gls{Waehler} \glslink{Benutzeroberflaeche}{GUI} und gibt den Namen \enquote{Anna Meier} und sein ursprüngliches \gls{Zertifikat} an.}
		{Die \glslink{Benutzeroberflaeche}{GUI} teilt ihm mit, das sein \gls{Zertifikat} ungültig für den gegeben Namen ist und bricht den Wahlvorgang ab.}

\test{Fehlerhafte Wahlkonfigurationen}{tst:wrongKonfig}
\tests{fnc:params}
\tests{fnc:dynamic-peers}

\teststep{\gls{Wahlleiter} \enquote{Hans Werner} hat seine GUI geöffnet um eine neue Wahl aufzusetzen}
		{Hans drückt den \enquote{Weiter} Button solange, bis der \enquote(Fertigstellen) Abschnitt erscheint.}
		{Der GUI-Bereich in dem normalerweise der Wahlname eingetragen ist, enthält eine Fehlermeldung die auf den fehlenden Wahlnamen hinweist.\\
		Neben der Liste der Kandidaten erscheint eine Fehlermeldung die auf die fehlenden Kandidaten hinweist.\\
		Neben der Liste der Wähler erscheint eine Fehlermeldung die auf die fehlenden Wähler hinweist.\\
		Der \enquote{Bestätigen} Button ist ausgegraut.}

\teststep{}
		{Hans drückt den \enquote{Allgemein} Button. Er gibt der Wahl den Namen \enquote{Beste Blockchain}. Er drückt daraufhin zweimal den \enquote{Weiter} Button.\\
		Hans fügt den drei Kandidaten hinzu. Der erste Kandidat bekommt den Namen \enquote{Bitcoin} und der zweite Kandidat den Namen \enquote{Cryptokitties}. Der letzte Kandidat bekommt keinen Namen.\\
		Zuletzt fügt Hans die vier Wähler \enquote{Martin Schleier}, \enquote{Peter Ente} und \enquote{Jürgen Gans} hinzu. Für den vierte Wähler trägt Hans keinen Namen ein. Hans drückt den \enquote{Weiter} Button.}
		{Neben der Liste der Kandidaten erscheint eine Fehlermeldung die darauf hinweißt, dass für einen Kandidat kein Name eingetragen wurde.\\
		Neben der Liste der Wähler erscheint eine Fehlermeldung die darauf hinweißt, dass für einen Wähler keinen Namen eingetragen wurde.}
	
\teststep{}
		{Hans drückt den \enquote{Kandidaten} Button. Er trägt den Namen \enquote{HyperLedger Fabric} für den dritten Kandidaten ein und drückt \enquote{Weiter}.\\
		Er trägt den Namen \enquote{Martina Storch} für den vierten Wähler ein und drückt \enquote{Weiter}.}
		{Es erscheint keine Fehlermeldung mehr und der \enquote(Bestätigen) Button ist anklickbar.}

\test{Import/Export Funktionalität}{tst:importExport}
\tests{fnc:import}
\tests{fnc:export}

\teststep{Wahlleiter Konrad Quack hat seine GUI gestartet und hat eine Wahl konfiguriert.}
		{Im \enquote{Fertigstellen} Abschnitt drückt er den \enquote{Exportieren} Button.}
		{Es erscheint eine Dialogbox zum Auswählen von Dateien.}

\teststep{}
		{Konrad wählt einen Speicherplatz.}
		{Eine Datei die die Wahlkonfiguration enthält wird an diesem Speicherplatz abgespeichert.}

\teststep{Konrad hat seine GUI erneut eröffnet.}
		{Er wählt den \enquote{Importieren} Button.}
		{Es erschein eine Dialogbox zum Auswählen von Dateien.}
		
\teststep{}
		{Konrad wählt die Wahlkonfigurationsdatei aus.}
		{Die Konfigurations-GUI öffnet sich, mit den Einstellungen aus der Datei in die GUI eingetragen.}

\section{Systemmodelle}

\subsection{Use-Case-Diagramm}
\begin{figure}[H]
	\fbox{\includegraphics[width=\textwidth]{res/UseCaseDiagram.png}}
	\caption{\label{fig:usecase}
		Ein Use-Case-Diagramm mit den typischen Funktionen der Software.
	}
\end{figure}

\subsection{Netzwerk}
\begin{figure}[H]
	\fbox{\includegraphics[width=\textwidth]{res/netzwerk_topologie.jpg}}
	\caption{\label{fig:network}
		Darstellung des Blockchain Netzwerkes.
	}
\end{figure}

\subsection{Systemarchitektur}
\begin{figure}[H]
	\fbox{\includegraphics[width=\textwidth]{res/systemArchitektur.png}}
	\caption{\label{fig:sysArch}
		Veranschaulichung der vorgesehenen Systemarchitektur.
	}
\end{figure}

\section{Benutzeroberfläche}

\subsection{Wahlleiter}

\begin{figure}[H]
	\fbox{\includegraphics[width=\textwidth]{res/wahlleiter_start.jpg}}
	\caption{\label{fig:wlltr-start}
		Die Startseite der \gls{Wahlleiter} \gls{Benutzeroberflaeche};
	}
\end{figure}
Der \gls{Wahlleiter} kann über den \enquote{Wahlen importieren} Button eine bereits \glslink{Konfiguration}{konfigurierte} \gls{Wahl} laden oder über \enquote(Neue Wahlen erstellen) eine neue \gls{Wahl} \glslink{Konfiguration}{konfigurieren}. 

\begin{figure}[H]
	\fbox{\includegraphics[width=\textwidth]{res/wahlleiter_allgemein.jpg}}
	\caption{\label{fig:wlltr-general}
		Allgemeine Einstellungen einer \gls{Wahl}.
	}
\end{figure}
Der \gls{Wahlleiter} kann einen Namen für die \gls{Wahl} eintragen und das zu verwendende \gls{Auszaehlungsverfahren}{Wahlsystem} in einem Dropdown-Menü auswählen.
Er kann der \gls{Wahl} eine Beschreibung hinzufügen, es wird aber nicht vorausgesetzt um die Wahl zu starten.
Der \enquote{Abbrechen} Button beendet den Konfiguration, verwirft alle Einstellungen und schließt das \gls{Konfigurationsmenu}.
Bei betätigen des \enquote{Weiter} Buttons wechselt er zum nächsten Schritt der Wahlkonfiguration.
%Die Angabe eines Namens und Wahlsystems sind hierbei notwendig um fortzufahren.

\begin{figure}[H]
	\fbox{\includegraphics[width=\textwidth]{res/wahlleiter_zeitraum.jpg}}
	\caption{\label{fig:wlltr-time}
		Einstellungen um den Zeitraum der \gls{Wahl} zu bestimmen.
	}
\end{figure}
Der \gls{Wahlleiter} kann den Anfang und das Ende des Wahlvorgangs bestimmen. Dabei kann er das Datum und die Uhrzeit wählen. Um die \gls{Wahl} direkt nach der \gls{Konfiguration} zu starten, kann der \gls{Wahlleiter} den \enquote{sofort} Button drücken. Die Anfangs Eingabeflächen werden daraufhin auf das derzeitige Datum und Uhrzeit gesetzt.
In dem Dropdown-Menü \enquote{Extrabedingung} kann er eine alternative Endbedingung für die \gls{Wahl} auswählen.
Abhängig davon welche der \gls{Wahlleiter} wählt, erscheinen unter dem Dropdown-Menü noch zusätzliche Optionen für die ausgewählte Extrabedingung.
%Sowohl das Anfangs als auch das Ende des Wahlvorgangs müssen angegeben sein um fortzufahren.

\begin{figure}[H]
	\fbox{\includegraphics[width=\textwidth]{res/wahlleiter_kandidaten.jpg}}
	\caption{\label{fig:wlltr-candidate}
		Hinzufügen der \glslink{Kandidat}{Kandidaten}.
	}
\end{figure}
Der \gls{Wahlleiter} kann über den \enquote{+} Button einen neuen \glspl{Kandidat} hinzufügen. Auf der \gls{Benutzeroberflaeche} erscheint daraufhin eine neue Zeile. In dem Eingabefeld wird der Name des \glslink{Kandidat}{Kandidaten} eingegeben. Der \enquote{Beschreibung} Button ermöglicht es, dem \glslink{Kandidat}{Kandidaten} eine optionale Beschreibung zu geben. Dieser Button öffnet ein neues Fenster mit einem Textfeld in das die Beschreibung eingegeben werden kann.
Der \enquote{-} Button in jeder Zeile löscht den jeweiligen \glspl{Kandidat}.

%Jeder Kandidat benötigt einen angegeben Namen um fortzufahren. Angebung einer Beschreibung ist optional.

\begin{figure}[H]
	\fbox{\includegraphics[width=\textwidth]{res/wahlleiter_waehler.jpg}}
	\caption{\label{fig:wlltr-voter}
		Hinzufügen der \gls{Waehler}.
	}
\end{figure}
Der \gls{Wahlleiter} kann nach gleichem Prinzip \gls{Waehler} hinzufügen und entfernen.
Er kann hierbei nur die Namen der \gls{Waehler} angeben.

\begin{figure}[H]
	\fbox{\includegraphics[width=\textwidth]{res/wahlleiter_fertig.jpg}}
	\caption{\label{fig:wlltr-done}
		Übersicht über alle Einstellungen.
	}
\end{figure}
Der Wahlleiter sieht alle vorgenommenen Einstellungen seiner Wahl:
\begin{enumerate}
	\item Den Start- und Endzeitpunkt der \gls{Wahl}.
	\item Die optionale zusätzliche Endbedingung der \gls{Wahl}.
	\item Alle zur \gls{Wahl} verfügbar stehenden Kandidaten inkl. Name und Beschreibung.
	\item Alle zur \gls{Wahl} berechtigten \gls{Waehler} und deren Namen.
\end{enumerate}
Bei betätigen des ''Export'' Buttons kann der \gls{Wahlleiter} die \gls{Konfiguration} der \gls{Wahl} speichern.
Bei betätigen des ''Bestätigen'' Buttons wird die \gls{Wahl} erstellt.

\begin{figure}[H]
	\fbox{\includegraphics[width=\textwidth]{res/wahlleiter_status.jpg}}
	\caption{\label{fig:wlltr-status}
		Übersicht über den aktuellen Wahlstand.
	}
\end{figure}
Der \gls{Wahlleiter} bekommt während dem Ablauf der \gls{Wahl} tabellarisch und in Form eines Diagrammes den aktuellen Stand seiner \gls{Wahl} dargestellt. Über den \enquote{Allgemeine Info.} Button wird die Beschreibung der \gls{Wahl} anstatt des Diagrammes angezeigt.


\subsection{Wähler}

\begin{figure}[H]
	\fbox{\includegraphics[width=\textwidth]{res/waehler_start.jpg}}
	\caption{\label{fig:whlr-start}
		Anfangs-GUI die das \gls{Zertifikat} des Wählers anfragt.
	}
\end{figure}
Der \gls{Waehler} kann sein \gls{Zertifikat} auswählen indem er den Button mit dem Pfeil nach oben drückt. Es erscheint ein Dateibrowser mit dem er seine Zertifikats-Datei auswählt. Er bestätigt seine Auswahl bei betätigen des \enquote{Weiter} Buttons. Bei erfolgreicher Authentifizierung wird er zur \gls{Wahl} weiter geleitet.

\begin{figure}[H]
	\fbox{\includegraphics[width=\textwidth]{res/waehler_wahl.jpg}}
	\caption{\label{fig:whlr-wahl}
		Auswahl der Kandidaten.
	}
\end{figure}
Der \gls{Waehler} kann einen oder mehrere \glspl{Kandidat} auswählen und seine Auswahl beliebig oft ändern.
Das anwählen des ''i'' Buttons zeigt die Beschreibung des entsprechenden \glspl{Kandidat} im rechten, leeren Bereich der \gls{Benutzeroberflaeche}.
Der \enquote{Allgemeine Info.} Button zeigt die Beschreibung der \gls{Wahl} im rechten, leeren Bereich der \gls{Benutzeroberflaeche} an. Wenn dieser Bereich schon mit einer anderen Beschreibung gefüllt ist, wird diese ersetzt.
Bei betätigen des ''OK'' Buttons wird die \gls{Wahl} bestätigt.


\begin{figure}[H]
	\fbox{\includegraphics[width=\textwidth]{res/waehler_result.jpg}}
	\caption{\label{fig:whlr-result}
		Ergebnisse der \gls{Wahl}.
	}
\end{figure}
Der \gls{Waehler} kann seine abgegebene Auswahl einsehen.
Sobald der Wahlvorgang \glslink{Wahl-Ende}{beendet} wird sieht er zusätzlich die Ergebnisse der \gls{Wahl} in der Form vom Prozenten und eines Diagramms.
Der \enquote{Allgemeine Info.} Button zeigt die Beschreibung der \gls{Wahl} an der Stelle des Diagramms.
Durch betätigen des \enquote{Ende} Buttons schließt sich die \gls{Benutzeroberflaeche} des \glslink{Waehler}{Wählers}.

\section{Spezielle Anforderungen an die Entwicklungsumgebung}
Zur Entwicklung werden die IntelliJ oder Eclipse IDE verwendet.
Zudem wird ein \gls{Linux} basiertes Betriebssystem verwendet.

\section{Zeit- und Ressourcenplanung}
Das System lässt sich grob in drei Schichten \hyperref[fig:sysArch]{aufgliedern}: \gls{Benutzeroberflaeche}, \gls{Klient} und \gls{Netzwerk}. Hierfür sind folgende Verantwortlichen vorgesehen:\\
\gls{Benutzeroberflaeche} - Achim\\
\gls{Klient} - David und Artem\\
\gls{Netzwerk} - Philipp und Tim\\
\\
Die zeitliche Einteilung ist wie folgt vorgesehen:\\
01.06. - 27.06. Entwurf und Entwicklung eines Prototypen.\\
29.06. - 25.07. Implementierung\\
10.08. - 29.08. Qualitätssicherung\\
05.09. Interne Abnahme

\section{Glossar}
\printglossaries

\end{document}
