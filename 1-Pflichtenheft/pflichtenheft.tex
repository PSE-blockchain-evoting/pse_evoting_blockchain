\documentclass[parskip=full,11pt,twoside]{scrartcl}
\usepackage[utf8]{inputenc}

\title{Blockchain-basiertes E-Voting}
\author{TODO}

% section numbers in margins:
\renewcommand\sectionlinesformat[4]{\makebox[0pt][r]{#3}#4}

% header & footer
\usepackage{scrlayer-scrpage}
\lofoot{\today}
\refoot{\today}
\pagestyle{scrheadings}

\usepackage[sfdefault,light]{roboto}
\usepackage[T1]{fontenc}
\usepackage[german]{babel}
\usepackage[yyyymmdd]{datetime} % must be after babel
\renewcommand{\dateseparator}{-} % ISO8601 date format
\usepackage{hyperref}
\usepackage{amsmath} % for $\text{}$
\usepackage[nameinlink]{cleveref}
\crefname{figure}{Abb}{Abb}
\usepackage[section]{placeins}
\usepackage{xcolor}
\usepackage{graphicx}
\hypersetup{
	pdftitle={Pflichtenheft},
	bookmarks=true,
}
\usepackage{csquotes}

\usepackage{pflichtenheft}

\begin{document}
\maketitle

\pagebreak
%%%%%%%%%%%%%%
\section{Produktübersicht}
Ziel des Projektes ist die Bildung eines (verlaesslichen und sicheren) Wahlsystems.  Das System soll Kosten minimieren, gleichzeitig Sicherheit und Transparenz der Wahlen gewaehrleisten. Kostenminimierung wird mit Anwendung von elektronischen Wahlen erreicht. Die Block-Chain-Technologie garantiert Wahlsicherheit.
\section{Zielbestimmung}

\subsection{Musskriterien}

\criterium{Stimmabgabe}{crt:length}
Jeder Wähler muss einen Kandidaten wählen können.

\criterium{Erstellung von Wahl}{crt:length}
Der Wahlleiter muss eine Wahl erstellen können

\criterium{Stimmenübermittlung}{crt:length}
Die Übermittlung einer Stimme muss öffentlich erfolgen und nachverfolgbar sein.

\criterium{Stimmenauszählung}{crt:length}
Die Auszählung der Stimmen erfolgt automatisch auf Anfrage des Wahlleiters.

\criterium{Graphische Benutzeroberfläche}{crt:length}
Es muss zwei getrennte graphische Benutzerberflächen geben. Die Benutzeroberfläche des Wahlleiters erlaubt es eine Wahl zu starten, zu beenden und automatisiert auszuzählen. Die Benutzeroberfläche des Wählers erlaubt es für einen der gegebenen Kandidaten zu stimmen und bietet Erklärungen zum Ablauf der Wahl. 

\criterium{Software-Architektur}{crt:length}
Die Software muss klar in Komponente aufgeteilt sein. Koponenten sind nur  wohldefinierte Schnittstellen verbunden.
Kritische und unkritische Komponenten sind klar erkennbar.

\subsection{Sollkriterien}

\subsection{Kannkriterien}

\criteriumOptional{Weitere Auszählverfahren}{crt:tally}
Bereitstellung von verschiedenen Auszählverfahren wie First-past-the-post, Instant-Runoff-Voting

\criteriumOptional{Geheime Wahlen}{crt:secret}
Bereitstellung von Wahlen bei denen die Stimme einzelner Teilnehmer nicht öffentlich ersichtlich sind

\criteriumOptional{Verteilung von Zugangsdaten}{crt:certdist}
Funktionalität die es dem Wahlleiter erlaubt Zugangsdaten für die Wahl per Email an die Wähler zu verteilen

\criteriumOptional{Automatische Wahlende}{crt:autoend}
Einstellbarkeit einer Endbedingung deren Erfüllung die Wahl automatisch beendet \\(z.B: 75\% aller registrierten Teilnehmer müssen abgestimmt haben)

\subsection{Abgrenzungskriterien}

%%%%%%%%%%%
\section{Produkteinsatz}

\subsection{Anwendungsbereiche}
Verwaltung der Wahlen im Rahmen der Organisation, der Stadt, des Landes.

\subsection{Zielgruppen}
\begin{enumerate}
  \item Wähler.
  \item Netzwerkadministratoren.
\end{enumerate}

\subsection{Abgrenzungskriterien}
\begin{enumerate}
  \item Das Produkt muss mit minimalen Netzwerkadministration zu betreiben sein.
  \item Wähler soll keine besondere Fähigkeiten und Wissen  haben, um das System nutzen zu können.
  \item Wähler muss per Netz wählen. Dazu braucht man den Internet-Zugang.
\end{enumerate}

\section{Produktumgebung}

\section{Funktionale Anforderungen}

\functionality{Schnelle Weiterleitung}{fnc:o1}
Um zu einer Kurz- die Lang-URLs zu finden,
benutzt der Dienst einen $O(1)$ Mechanismus.
Es wird sichergestellt,
dass mit einer zunehmenden Anzahl von Kurz-URLs im System
das Finden nicht länger dauert.

\section{Produktdaten}

\section{Produktleistungen}


\section{Weiter Nicht-Funktionale Anforderungen}

\nonFunctionality{Modernes Design}{nfc:design}

Das Design soll modern und seriös wirken.


\section{Qualitätsanforderungen}
\begin{enumerate}
	\item \textbf{Korrektheit der Wahlergebnisse}: Jede gültige Stimme ist garantiert im Wahlergebnis enthalten.
	\item \textbf{Protokollierung des Netzwerkes}: Ereignisse und Probleme auf dem Blockchain Netzwerk, werden übersichtlich in einer Logdatei protokolliert. Diese Logdatei ist für den Wahlleiter einsehbar.
	\item \textbf{Unveränderbarkeit der Wahl}: Sobald die Wahl vom Wahlleiter einmal aufgesetzt wurde, können die Eigenschaften dieser Wahl von niemandem (insbesondere vom Wahlleiter) mehr verändert werden.
	\item \textbf{Vermeidung von sinnlosen Eigenschaften der Wahl}: Während dem Aufsetzen der Wahl wird der Wahlleiter auf eventuelle Fehler in der Konfiguration hingewiesen. (z.B Nur eine Wahlmöglichkeit, Fehlende Wähler, triviale Endbedingung der Wahl)
	\item \textbf{Verhinderung des Double-Spending-Problems}: Jeder Wähler kann nur einmal den Wahlvorgang erfolgreich durchlaufen. Bei einem zweiten Versuch kommt es zu einer entsprechenden Fehlermeldung auf der Wähler GUI.
	%Ich rede hier extra vom Wahlvorgang anstatt von Stimmen, da es bei manchen Wahlverfahren vorkommen könnte, das ein Wähler mehrere Stimmen hat.
	\item \textbf{Vermeidung von ungewollten Enthaltungen}: Falls ein Wähler seine Wahl bestätigen möchte ohne das dieser für einer der Wahlmöglichkeiten gestimmt hat, wird eine Warnung angezeigt.
	\item \textbf{Warnungen bei Netzwerkproblemen}: Bei Problemen die Stimme eines Wählers zu übertragen, wird dieser Wähler daraufhin informiert und die Möglichkeit geboten seine Stimme erneut zu übertragen. 
\end{enumerate}

\section{Globale Testfälle}
\test{Ablauf einer Wahl}{}

\teststep{Wahlleiter \enquote{Fritz Müller} hat die GUI geöffnet um eine neue Wahl aufzusetzen.}
		{Fritz gibt den Namen \enquote{Vorstandswahl 2018} an und fügt eine passende Beschreibung hinzu.\\
		Fritz fügt \enquote{Max Mustermann}, \enquote{Anna Meier}, \enquote{Erich Schmitt} und 10 andere Wähler hinzu.\\
		Zuletzt werden \enquote{Johannes Heinzhof}, \enquote{Wolfgang Rudolf} und \enquote{Sabine Scholl} als Wahlmöglichkeiten von Fritz festgelegt,}
		{Die Zertifikate für die alle zugelassenen Wähler werden generiert.}
		
\teststep{}
		{Fritz wählt aus, dass die Wahl am 1. August 2018 beginnt und am 31. August 2018 endet. Er bestätigt seine Eingaben.}
		{Die Wahl ist jetzt im Blockchain Netzwerk aktiv.}

\teststep{Max Mustermann startet die Wähler GUI am 1. August 2018}
		{Max gibt seinen Namen und sein Authentifizierungs Zertifikat an.}
		{Die GUI updatet sich und zeigt alle Wahlmöglichkeiten an}
		
\teststep{}
		{Max wählt \enquote{Sabine Scholl} aus den Wahlmöglichkeiten aus und bestätigt seine Wahl.}
		{Die GUI schickt seine Stimme ab und informiert ihn dass diese erfolgreich gezählt wurde.}
		
\teststep{Der 31. August 2018 ist erreicht.}
		{Die Wahl beendet sich auf dem Blockchain Netzwerk.}
		{Die Wahlergebnisse können auf den GUIs eingesehen werden.}
		
\test{Verhinderung von Angriffsversuchen auf die Wahl}{}

\teststep{Wiederholung des Schritte von T1 1.1-1.4}
		{Max startet erneut eine Wähler GUI und gibt seine Namen und Zertifikat an.}
		{Die GUI teilt ihm mit, dass seine Stimme schon gegeben wurde und bricht den Wahlvorgang ab.}

\teststep{}
		{Max startet erneut eine Wähler GUI und gibt den Namen \enquote{Anna Meier} und sein ursprüngliches Zertifikat an.}
		{Die GUI teilt ihm mit, das sein Zertifikat ungültig für den gegeben Namen ist und bricht den Wahlvorgang ab.}

\section{Spezielle Anforderungen an die Entwicklungsumgebung}
Zur Entwicklung werden die IntelliJ oder Eclipse IDE verwendet.
Zudem wird ein Linux basiertes Betriebssystem verwendet.

\section{Glossar}

\textbf{Besucher}:
Eine Person, welche den Dienst nutzt.
Kann eingeloggt sein oder nicht.

\textbf{Dienst}:
Die Software im laufenden Betrieb. Software as a Service.

\textbf{Homepage}:
Seite, die beim Besuchen der Betreiberdomain \emph{ohne Pfad} angezeigt wird. Auch \enquote{Startseite}.

\textbf{First-past-the-post}:
Auszählungsverfahren, bei dem der Kandidat mit den meisten Stimmen gewinnt

\textbf{Instant-Runoff-Voting}:
Auszählungsverfahren, bei dem der Wähler die Kandidaten nach Präferenz ordnet.

\end{document}
