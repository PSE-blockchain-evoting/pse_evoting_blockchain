\newglossaryentry{Benutzer}{name={Benutzer},description={
	Eine Person die mit der Software interagiert.
}}
	
	\newglossaryentry{Wahl}{name={Wahl},description={
	Eine Sammlung von Stimmen für die Bestimmung eines Kandidaten.
}}
		
\newglossaryentry{Waehler}{name={Wähler},description={
	Person die bei einer Wahl einen bestimmten Kandidaten wählt und die hierfür nötigen Berechtigungen hat.
}}
			
\newglossaryentry{Kandidat}{name={Kandidat},description={
	Eine Entität, für die ein Wähler bei einer Wahl \glslink{Stimme}{stimmen} kann.
	Dies Beinhaltet Personen, Parteien und andere beliebige Auswahlmöglichkeiten.
}}
				
\newglossaryentry{Stimme}{name={Stimme},description={
	Eine Zähleinheit, die im \gls{Auszaehlungsverfahren} zur Ermittlung des Wahlergebnisses benutzt wird.
}}

\newglossaryentry{Auszaehlungsverfahren}{name={Auszählungsverfahren},description={
	Die Methode um zu bestimmen, welcher Kandidat bei einer Wahl gewonnen hat.
}}

\newglossaryentry{Absolute-Mehrheitswahl}{name={Absolute Mehrheitswahl},description={
Auszählungsverfahren, bei dem der Kandidat gewinnt, der über 50\% der Stimmen hat. Ansonsten hat niemand gewonnen.
}}

\newglossaryentry{Relative-Mehrheitswahl}{name={Relative Mehrheitswahl},description={
Auszählungsverfahren, bei dem der Kandidat mit den meisten Stimmen gewinnt.
}}

\newglossaryentry{Instant-Runoff-Voting}{name={Instant-Runoff-Voting},description={
Auszählungsverfahren, bei dem der Wähler die Kandidaten nach Präferenz ordnet. Sei n die Anzahl der zur Auswahl stehenden Kandidaten, das Verfahren ist für Wahlen mit 3 oder mehr Kandidaten geeignet:
\begin{enumerate}
	\item Jeder Wähler kann jedem Kandidaten einen Wert von 1 bis n zuweisen. Werte düfen nicht mehrfach vergeben werden. Kandidaten müssen nicht bewertet werden.
	\item Bei der Auszählung wird bestimmt welcher Kandidat die wenigsten Stimmen mit Wert 1 bekommen hat. Dieser Kandidat wird dann aus allen Listen entfernt. Die Werte der Kandidaten auf den nachfolgenden Plätzen werden jeweils um 1 abgezogen.
	\item Schritt 2 wird so lange wiederholt bis nur noch 2 Kandidaten übrig sind. Der Kandidat mit den meisten ersten Stimmen hat gewonnen.
\end{enumerate}
}}

\newglossaryentry{Benutzeroberflaeche}{name={Benutzeroberfläche},description={
Grafische Benutzerschnittstelle zu einem Computer, welche eine leichte und intuitiver Benutzung der Software ermöglicht.
}}

\newglossaryentry{Konfigurationsmenu}{name={Konfigurationsmenu},description={
Die \gls{Benutzeroberflaeche} des \glslink{Wahlleiter}{Wahlleiters} in der er eine \gls{Wahl} \glslink{Konfiguration}{konfigurieren} kann.
}}

\newglossaryentry{Zertifikat}{name={Zertifikat}, plural=Zertifikate,description={
Digitaler Datensatz, der die Authentizität und Integrität von Personen oder Objekten durch kryptografische Verfahren nachweist.
}}

\newglossaryentry{Blockchain}{name={Blockchain},description={
Datenbanktechnologie bei der die einzelnen Datensegmente verknüpft werden um so Eigenschaften wie Unveränderbarkeit und Dezentralisierung zu erreichen.
}}

\newglossaryentry{Ledger}{name={Blockchain-Ledger},description={
Datenbank in einer Blockchain, welche alle Transaktionen, die auf der Blockchain ausgeführt wurden, enthält. In dem Kontext unserer Wahl sind die Transaktionen einzelne Stimmen. Jeder Peer in dem Netzwerk besitzt einen Ledger.
}}

\newglossaryentry{Netzwerk}{name={Blockchain-Netzwerk},description={
Menge an verknüpften \glslink{Peer}{Peers}, die zusammen einen synchronierten \gls{Ledger} verwalten.
}}

\newglossaryentry{Peer}{name={Peer},description={
Computer der eine Kopie des Ledgers enthält, diese mit anderen Peers synchronisiert und eine Schnittstelle zwischen Chaincode und Software bereitstellt.
}}

\newglossaryentry{Chaincode}{name={Chaincode},description={
Programme die auf der Blockchain laufen und als Schnittstelle eines Peers zur Blockchain dienen.
}}

\newglossaryentry{Logdatei}{name={Logdatei},description={
Datei welche alle Ereignisse (Informationen, Warnungen und Fehler) in einem System protokolliert.
}}

\newglossaryentry{Linux}{name={Linux},description={
Beliebtes Unix-like Betriebssystem.
}}

\newglossaryentry{Double-Spending-Problem}{name={Double-Spending-Problem},description={
Die mehrfache, erfolgreiche Abgabe einer Stimme in einem Wahlsystem. Die erfolgreiche \gls{Stimmabgabe} darf pro \gls{Waehler} maximal einmal erfolgen.
}}

\newglossaryentry{Wahlleiter}{name={Wahlleiter},description={
Administrator der Wahl, legt die Einstellungen, Wähler und Kandidaten der Wahl fest.
}}

\newglossaryentry{Konfiguration}{name={Konfiguration},description={
Feste Belegung der einstellbaren Eigenschaften eines Systems (hier i.d.R die Wahl).
}}

\newglossaryentry{Benutzerdaten}{name={Benutzerdaten},description={
Daten über den Benutzer. (z.B Name, Alter, Geburtsdatum und Gewohnheiten)
}}

\newglossaryentry{Wahlberechtigung}{name={Wahlberechtigt},description={
Vom Wahlleiter als Wähler zugelassen. (BESPRECHEN)
}}

\newglossaryentry{Stimmabgabe}{name={Stimmabgabe},description={
Sobald ein Wähler in der GUI seine Stimme gewählt hat bestätigt er diese. Seine Stimme wird dann von der GUI an das Blockchain-Netzwerk gesendet. Sofern die Stimme gültig ist wird sie in den Ledger eingetragen. Ansonsten wird die Stimme verworfen. Eine Stimme ist genau dann gültig wenn der Wähler sich als Wahlberechtigt authentifizieren konnte.
}}

\newglossaryentry{Stimmenauszaehlung}{name={Stimmenauszählung},description={
Das Auslesen der Stimmen aus der Blockchain um zu bestimmen wer die Wahl gewonnen hat. Dabei wird beachtet welches Wahlverfahren für die Wahl definiert wurde.
}}

\newglossaryentry{Wahl-Ende}{name={Wahl-Ende},description={
Sobald die in der Wahl Konfiguration eingestellte Endbedingung erfüllt ist, beendet sich die Wahl. Es ist nicht mehr möglich gültige Stimmen abzugeben. Die Ergebnisse der Wahl werden für jeden Wähler auf dessen Klienten ausgewertet und in der GUI angezeigt.
}}

\newglossaryentry{Klient}{name={Klient},description={
Das Programm das auf den Computern der Wählern und des Wahlleiter läuft. Es stellt die GUI bereit und verwaltet die Kommunikation mit dem \gls{Netzwerk}.
}}

\newglossaryentry{Wahlstand}{name={Wahlstand},description={
		Das Auswertungsergebnis der Wahl, bei dem alle bisher erfolgreich abgegebenen Stimmen ausgezählt werden.
}}

